\documentclass{article}

% if you need to pass options to natbib, use, e.g.:
\PassOptionsToPackage{numbers, compress}{natbib}
% before loading neurips_2021
% ready for submission
\usepackage[final]{neurips_2021}

% to compile a preprint version, e.g., for submission to arXiv, add add the
% [preprint] option:
%     \usepackage[preprint]{neurips_2021}

% to compile a camera-ready version, add the [final] option, e.g.:
%     \usepackage[final]{neurips_2021}

% to avoid loading the natbib package, add option nonatbib:
%    \usepackage[nonatbib]{neurips_2021}

\usepackage[utf8]{inputenc} % allow utf-8 input
\usepackage[T1]{fontenc}    % use 8-bit T1 fonts
\usepackage{hyperref}       % hyperlinks
\usepackage{url}            % simple URL typesetting
\usepackage{booktabs}       % professional-quality tables
\usepackage{amsfonts}       % blackboard math symbols
\usepackage{nicefrac}       % compact symbols for 1/2, etc.
\usepackage{microtype}      % microtypography
\usepackage{xcolor}         % colors

\usepackage{amsmath}
\usepackage{bm}
\usepackage{bbm}

\usepackage{algorithm}
\usepackage{algorithmic}
% \usepackage{algpseudocode}
\usepackage{amsthm}
\usepackage{mathtools}

\usepackage{subfigure}

\usepackage{wrapfig}

% \newcommand{\tu}[1]{{\underline{\textit{#1}}}}
% \newcommand{\tbf}[1]{{\underline{\textbf{#1}}}}
\newcommand{\tu}{\textbf}
\newcommand{\tbf}{\textbf}
\usepackage{color}
\newcommand{\SC}[1]{{\color{black}#1}}
\newcommand{\JC}[1]{{\color{black}#1}}
\newcommand{\expec}{\mathbb{E}}
\newcommand{\reals}{\mathbb{R}}
\usepackage{balance}
%\usepackage{breakurl}
%\AtBeginEnvironment{proof}{\renewcommand{\qedsymbol}{}}{}{}

%\setlength{\dbltextfloatsep}{0.5pt}

%\usepackage{etoolbox}
%\makeatletter
%\patchcmd\@combinedblfloats{\box\@outputbox}{\unvbox\@outputbox}{}{\errmessage{\noexpand patch failed}}
%\makeatother
%
\usepackage{hyperref}
\def\UrlBreaks{\do\/\do-}

%\usepackage[font=small]{caption}
\usepackage{caption}

%\usepackage{algorithmic,comment}
%\usepackage{algorithm} 
%\usepackage{algorithmic,comment}  
%\usepackage[algo2e,ruled,vlined,commentsnumbered,titlenotnumbered,linesnumbered]{algorithm2e}
%\usepackage{pseudocode} 
\usepackage[algo2e]{algorithm2e} 



\newcommand{\Nepochs}{N_{\text{epoch}}}
%%%%%%%%%%%%%%%%%%%%%%%%%%%%%%%%%%%%%%%%
\newcommand{\argmin}{\mathop{\rm argmin}}
\newcommand{\argmax}{\mathop{\rm argmax}}

\newcommand{\eg}{{\it e.g.}}
\newcommand{\ie}{{\it i.e.}}
\newcommand{\cf}{{\it cf.}}
\newcommand{\etal}{{\it et al.}}

% with MPC
%\newcommand{\bolduMPCopt}{\bm{u_t}^{\text{MPC},*}}
%\newcommand{\bolduMPChat}{\bm{\hat{u}_t}^{\text{MPC},*}}
%\newcommand{\uMPCopt}{{u_t}^{\text{MPC},*}}
%\newcommand{\uMPChat}{{\hat{u}_t}^{\text{MPC},*}}

%\newcommand{\bolduMPCopt}{\bm{u_t}^{*}}
% \newcommand{\bolduMPCopt}{\bm{u}^{*}}
% \newcommand{\bolduMPChat}{\bm{\hat{u}}}
%\newcommand{\bolduMPChat}{\bm{\hat{u}_t}}
\newcommand{\bolduMPCopt}{\boldu^{*}}
\newcommand{\bolduMPChat}{\bolduhat}
\newcommand{\uMPCopt}{u_t^{*}}
\newcommand{\uMPChat}{{\hat{u}_t}}

\newcommand{\boldw}{\bold{w}}

% definitions
\newtheorem{definition}{Definition}
\newtheorem{remark}{Remark}
\newtheorem{thm}{Theorem}
\newtheorem{clm}{Claim}
\newtheorem{prop}{Proposition}
\newtheorem{examp}{Example}
\newtheorem{lem}{Lemma}
\newtheorem{cor}{Corollary}
\newtheorem{problem}{Problem}

% task machine perception
%\newcommand{\Aencoder}{A}
%\newcommand{\Bdecoder}{B}
%\newcommand{\Atilde}{\tilde{A}}
%\newcommand{\Btilde}{\tilde{B}}
%\newcommand{\KLQR}{K}
%\newcommand{\taskoperand}{\KLQR x - \KLQR \Bdecoder \Aencoder x}
%\newcommand{\reconoperand}{x - \Bdecoder \Aencoder x}

\newcommand{\control}{\mathrm{c}}
\newcommand{\encoder}{\mathrm{e}}
\newcommand{\decoder}{\mathrm{d}}
\newcommand{\forecaster}{\mathrm{F}}

\newcommand{\thetacontrol}{\theta_\control}
\newcommand{\thetaencoder}{\theta_\encoder}
\newcommand{\thetadecoder}{\theta_\decoder}
\newcommand{\ftask}{f}
\newcommand{\shat}{\hat{s}}
\newcommand{\zbottleneck}{Z}
\newcommand{\ninput}{n}
\newcommand{\moutput}{m}
\newcommand{\nH}{nH}
\newcommand{\mH}{mH}
\newcommand{\pH}{pH}
\newcommand{\fencoder}{p}
\newcommand{\fdecoder}{q}
\newcommand{\bolds}{\bold{s}}
\newcommand{\boldS}{\bold{S}}
\newcommand{\boldShat}{\bold{\hat{S}}}
\newcommand{\boldshat}{\bold{\shat}}
\newcommand{\boldu}{\bold{u}}
\newcommand{\bolduhat}{\bold{\hat{u}}}
\newcommand{\boldx}{\bold{x}}

\newcommand{\boldst}{\bold{s_t}}
\newcommand{\boldshatt}{\bold{\shat_t}}

%\newcommand{\JMPC}{J^{\text{MPC}}}
\newcommand{\JMPC}{\Jcontrol}

\newcommand{\Jcontrol}{J^{\control}}
\newcommand{\Joverall}{J^{\mathrm{tot.}}}
\newcommand{\Jforecast}{J^{\forecaster}}
\newcommand{\lambdaforecast}{\lambda^{\forecaster}}

\newcommand{\transpose}{\intercal}
\newcommand{\optimal}{*}
\newcommand{\trans}{\top}
\newcommand{\sdim}{p}

% state, action, exogenous input dimensions
%%%%%%%%%%%%%%%%%%%%%%%%%%%%%%%%%%%%
\newcommand{\statedim}{\reals^{\ninput}}
\newcommand{\controldim}{\reals^{\moutput}}
\newcommand{\exoinputdim}{\reals^{\sdim}}
\newcommand{\doublevert}{\vert\vert}


\title{Data Sharing and Compression for Cooperative Networked Control}

% The \author macro works with any number of authors. There are two commands
% used to separate the names and addresses of multiple authors: \And and \AND.
%
% Using \And between authors leaves it to LaTeX to determine where to break the
% lines. Using \AND forces a line break at that point. So, if LaTeX puts 3 of 4
% authors names on the first line, and the last on the second line, try using
% \AND instead of \And before the third author name.


\usepackage{authblk}
\renewcommand*{\Authfont}{\bfseries} 
\author[1]{Jiangnan Cheng}
\author[2]{Marco Pavone}
\author[3]{Sachin Katti}
\author[4]{Sandeep Chinchali}
\author[1]{Ao Tang}

\affil[1]{School of Electrical and Computer Engineering, Cornell University, Ithaca, NY}
\affil[2]{Department of Aeronautics and Astronautics, Stanford University, Stanford, CA}
\affil[3]{Department of Computer Science, Stanford University, Stanford, CA}
\affil[4]{Department of Electrical and Computer Engineering, The University of Texas at Austin, Austin, TX \authorcr
  \{\tt jc3377, atang\}@cornell.edu, \{\tt pavone, skatti\}@stanford.edu, \tt sandeepc@utexas.edu}

% \author{%
%   Jiangnan Cheng\\
%   Cornell University\\
%   \And
%   Marco Pavone\\
%   Stanford University\\
%   \And
%   Sachin Katti\\
%   Stanford University\\
%   \AND
%   Sandeep Chinchali\\
%   Stanford University\\
%   \And
%   Ao Tang\\
%   Cornell University\\
% }

\begin{document}

\maketitle

\begin{abstract}
Sharing forecasts of network timeseries data, such as cellular or electricity load patterns, can improve independent control applications ranging from traffic scheduling to power generation. Typically, forecasts are designed without knowledge of a downstream controller's task objective, and thus simply optimize for \textit{mean} prediction error. However, such task-agnostic representations are often too large to stream over a communication network and do not emphasize salient temporal features for cooperative control. This paper presents a solution to learn succinct, highly-compressed forecasts that are \textit{co-designed} with a modular controller's task objective. Our simulations with real cellular, Internet-of-Things (IoT), and electricity load data show we can improve a model predictive controller's performance by at least $25\%$ while transmitting $80\%$ less data than the competing method. Further, we present theoretical compression results for a networked variant of the classical linear quadratic regulator (LQR) control problem.
\end{abstract}

\section{Introduction}
\label{sec:intro}
Cellular network and power grid operators measure rich timeseries data, such as city-wide mobility and electricity demand patterns. Sharing such data with \textit{external} entities, such as a
taxi fleet operator, can enhance a host of \SC{societal-scale control tasks}, ranging from
taxi routing to battery storage optimization. 
However, how should timeseries owners \textit{represent} their data to limit the scope and volume of information shared across a data boundary, such as a congested wireless network?\footnote{Uber processes petabytes of data per day \cite{Uber} and a mobile operator can process 60 TB of daily cell metrics \cite{SK}. Even a \textit{fraction} of such data is hard to send.}

At a first glance, it might seem sufficient to simply share generic demand forecasts with any downstream controller.
Each controller, however, often has a unique cost function and context-specific sensitivity to prediction errors.
%, which are often \textit{unknown} to the timeseries generator. 
For example, cell demand forecasts should emphasize accurate peak-hour forecasts for taxi fleet routing. The same underlying cellular data should instead emphasize fine-grained throughput forecasts when a video streaming controller starts a download. 
Despite the benefits of customizing forecasts for control, today's forecasts are mostly \textit{task-agnostic} and simply optimize for mean or median prediction error. As such, they often waste valuable network bandwidth to transmit temporal features that are unnecessary for a downstream controller. Even worse, they might not minimize errors when they matter most, such as peak-hour variability.
%Even worse, they often don't minimize error when it matters most, such as peak-hour variability.

% check
% \begin{figure}[t]
% \vskip 0.2in
% \begin{center}
% \centerline{\includegraphics[width=0.6\columnwidth]{figures/final_model_nabla_2.pdf}}
% \caption{\textbf{Data sharing for cooperative control:} An owner of timeseries data $s_t$, such as a mobile operator, needs to transmit a compressed representation $\phi_t$ to a downstream controller with internal state $x_t$. The \textit{learned} forecast emphasizes task-relevant temporal features to minimize end-to-end controller cost $J$.} 
% \label{fig_problem}
% \end{center}
% \vskip -0.4in
% \end{figure}

\begin{wrapfigure}{R}{0.5\columnwidth}
\centering
\includegraphics[width=0.5\columnwidth]{figures/final_model_nabla_2.pdf}
\caption{\textbf{Data sharing for cooperative control:} An owner of timeseries data $s_t$, such as a mobile operator, needs to transmit a compressed representation $\phi_t$ to a downstream controller with internal state $x_t$. The \textit{learned} forecast emphasizes task-relevant temporal features to minimize end-to-end controller cost $J$.}
\label{fig_problem}
% \vskip -0.5em
\end{wrapfigure}


Given the limitations of today's task-agnostic forecasts, this paper contributes a novel problem formulation for learning \textit{task-driven} forecasts for networked control. 
In our general problem (Fig. \ref{fig_problem}), an operator measures timeseries $s_t$, such as electricity or cell demand, and 
transmits compressed representation $\phi_t$, which is decoded to $\hat{s}_t$ at the controller. Rather than simply minimize the prediction error for $\hat{s}_t$, we instead learn a representation that minimizes a modular controller $\pi$'s ultimate cost $J$.
Our key technical insight is to compute a controller's sensitivity to prediction errors, which in turn guides how we \textit{co-design} and learn a concise forecast representation that is tailored to control. 
As such, our scheme jointly integrates data-driven forecasting, compression, and model-based control. 
%As such, we can jointly integrate data-driven forecasting with data compression and model-based control. 
%As such, we can blend data-driven forecasting with model-based control. 
%that is co-designed with a model-based controller.

%As such, we can interface a data-driven forecaster that is co-designed with the task objective of a model-based controller.
%by separating but \textit{co-designing} forecasting and control, we can blend model-based control with data-driven forecasters.
%Crucially, we can blend model-based control with data-driven forecasting by propagating a classical controller's sensitivity to forecasts to a modular forecaster's learned parameters.
%Morever, our approach allows us to blend neural network forecasters with model-based control by compart
%Our key technical insight is to \textit{co-design} and learn a compressed forecast representation tailored to a model-based controller, which allows us to blend data-driven forecasting and classical control. 
%techniques. 
%This insight allows us to learn representations that are $80\%$ smaller than those generated by standard autoencoders, even for real IoT sensor data we captured on embedded devices as well as standard benchmark datasets.
%Our key technical insight is to propagate a controller's sensitivity to prediction errors to guide learning of a modular forecasting model. As such, we can \textit{co-design} and learn a concise forecast representation that is tailored to a model-based controller, allowing us to seamlessly blend data-driven forecasting and classical control.
%Our key technical insight is to propagate a controller's sensitivity to prediction errors back to the modular forecasting model that generates such predictions. As such, we can \textit{co-design} and learn a concise forecast representation that is tailored to a model-based controller, allowing us to seamlessly blend data-driven forecasting and classical control.
%which in turn guides how we \textit{co-design} and learn a concise forecast representation tailored to control. As such, we can seamlessly blend classical model-based control with neural network forecasting models.

\tu{Related work:} 
Our work is broadly related to information-theoretic compression for control as well as task-driven representation learning. The closest work to ours is \cite{donti2017task}, where task-driven forecasts are learned for one-step stochastic optimization problems. In stark contrast, we address \textit{compression} of timeseries forecasts and focus on networked, multi-step control problems. Our work is also inspired by Shannon's rate-distortion theory \cite{berger2003rate}, which describes how to encode and transmit signals with a minimal bit-rate to minimize reconstruction error.
%\SC{In contrast, we observe that if we optimize for control cost, as opposed to simply reconstruction error, we can achieve
%significantly better compression gains by encoding only task-relevant features}.
%a forecast should only encode task-relevant features necessary to minimize control cost, which enables significant compression gains compared to optimizing for reconstruction error. 
In contrast, we work with real numbers rather than bits and focus on reducing the dimension of data while keeping task-specific control cost low.

%We observe that a forecast should only encode task-relevant features necessary to minimize control cost, which enables significant compression gains compared to optimizing for reconstruction error. Our distortion measure is task specific.

Prior work has addressed rate-distortion tradeoffs for networked LQR control problems \cite{kostina2019rate,tatikonda2004stochastic,schenato2007foundations}. However, these works focus on ensuring closed-loop stability for a remote controller and a physically-separated plant, \SC{such as in tele-operation}. Our problem is fundamentally different, since we address how \textit{external} timeseries forecasts can enhance a controller's \textit{local} decisions using full knowledge of its own internal state. 
\SC{While the term \textit{co-design} appears in select work on networked LQR, it refers to a drastically different setting where a communication scheduler and tele-operated controller must be jointly designed \cite{yun2011optimal,zhang2006communication,branicky2002scheduling,peng2013event}}. 
\JC{
Moreover, event-triggered control/learning \cite{heemels2012introduction,miskowicz2018event,schluter2020event,beuchert2020overcoming} emphases temporal sparsity of communications, while in our setting the MPC controller consistently requires a forecast of timeseries.
}
Finally, our work differs from deep neural network (DNN) compression schemes for video inference \cite{blau2019rethinking,DBLP:conf/rss/NakanoyaCADKP21} since we focus on control.
%that optimize for either human or machine perception tasks \cite{blau2019rethinking,nakanoya2020task} since we focus on multi-step control.


\JC{
More discussions will follow in Sec. \ref{sec:problem_statement} after the technical problem is introduced in detail.
}

\tu{Contributions:}
In light of prior work, our contributions are three-fold. First, we introduce a novel problem for learning compressed timeseries representations that are tailored to control. 
%Second, we illustrate how to use a controller's sensitivity to prediction errors to guide data-driven learning of concise, task-relevant forecasts.
%Second, we show how to compute a 
%controller's sensitivity to prediction errors, which guides learning of concise forecast representations as well as analytical compression results for linear control.
Second, to gain insights into our problem, we contribute analytic compression results for LQR control. These insights serve as a foundation for our general algorithm that computes the sensitivity of a model predictive controller (MPC) to prediction errors, which guides learning of concise forecast representations.
%Second, we contribute analytic compression results for LQR control, which serve as foundations for our model predictive control (MPC) experiments. Third, we illustrate the generality and efficacy of our approach using DNN forecasts for real cell, IoT, and electricity grid data. 
Third, we learn representations that improve control performance by $>25\%$ and are $80\%$ smaller than those generated by standard autoencoders, even for real IoT sensor data we captured on embedded devices as well as benchmark electricity and cell datasets.
%In particular, we collect timeseries from an embedded IoT sensor for our experiments.

\tu{Organization:} 
%This paper is organized as follows. 
In Sec. \ref{sec:problem_statement}, we formalize a general problem of compression for networked control and provide analytical results for LQR. Then, in Sec. \ref{sec:algorithm}, we contribute an algorithm for task-driven data compression for general MPC problems. We demonstrate strong empirical performance of our algorithm for cell, energy, and IoT applications in Sec. \ref{sec:scenarios} - \ref{sec:evaluation} and conclude in Sec. \ref{sec:conclusion}.

%We demonstrate strong empirical performance of our compression scheme for cellular traffic scheduling, energy forecasting, and control from IoT sensors in Sec. \ref{sec:scenarios} - \ref{sec:evaluation}. Finally, we conclude in Sec. \ref{sec:conclusion}.


\section{Problem Formulation}
\label{sec:problem_statement}
We now describe the information exchange between a generator of timeseries data, henceforth called a forecaster, and a controller, as shown in Fig. \ref{fig_problem}. Both systems operate in discrete time, indexed by $t$, for a time horizon of $T$ steps. The notation $y_{a:b}$ denotes a timeseries $y$ from time $a$ to $b$.

\tbf{Forecast Encoder:}
The forecaster measures a high-volume timeseries $s_t \in \reals^p$. Timeseries $s$ is drawn from a domain-specific distribution $\mathcal{D}$, such as cell-demand patterns, denoted by $s_{0:T-1} \sim \mathcal{D}$. A differentiable encoder maps the past $W$ measurements, denoted by $s_{t-W+1:t}$, to a compressed representation $\phi_t \in \reals^Z$, using model parameters $\thetaencoder$: $\phi_t = g_{\mathrm{encode}}(s_{t-W+1:t}; \thetaencoder)$. Typically, $Z\ll p$ and is referred to as the \textit{bottleneck} dimension since it limits the communication data-rate and how many floating-point values are sent per unit time.


\tbf{Forecast Decoder:}
The compressed representation $\phi_t$ is transmitted over a bandwidth-constrained communication network, where a downstream decoder maps $\phi_t$ to a forecast $\shat_{t:t+H-1}$ for the next $H$ steps, denoted by: $\hat{s}_{t:t+H-1} = g_{\mathrm{decode}}(\phi_t; \thetadecoder)$,  
where $\thetadecoder$ are decoder parameters. Importantly, we decode representation $\phi_t$ into a forecast $\hat{s}$ so it can be directly passed to a model-predictive controller that interprets $\hat{s}$ as a physical quantity, such as traffic demand. 
\SC{The encoder and decoder jointly enable compression and forecasting by mapping past observations to a forecast via bottleneck $\phi_t$}.


\tbf{Modular Controller:}
The controller has an internal state $x_t \in \reals^n$ and must choose an optimal control $u_t \in \reals^m$. We denote the admissible state and control sets by $\mathcal{X}$ and $\mathcal{U}$ respectively. The system dynamics also depend on external timeseries $s_t$ and are given by:
$x_{t+1} = f(x_t, u_t, s_t), \quad t \in \{0, \cdots, T-1\}$.
Importantly, while state $x_t$ depends on \textit{exogenous} input $s_t$, we assume $s_t$ evolves \textit{independently} of $x_t$ and $u_t$. This is a practical assumption in many networked settings. For example, the demand $s_t$ for taxis might mostly depend on city commute patterns and not an operator's routing decisions $u_t$ or fleet state $x_t$.
Ideally, control policy $\pi$ chooses a decision $u_t$ based on fully-observed internal state $x_t$ and \textit{perfect} knowledge of exogenous input $s_{t:t+H-1}$:
$u_t = \pi(x_t, s_{t:t+H-1}; \thetacontrol)$, 
where $\thetacontrol$ are control policy parameters, such as a feedback matrix for LQR. However, in practice, given a possibly noisy forecast $\hat{s}_{t:t+H-1}$, it will \textit{enact} a control denoted by $\hat{u}_t = \pi(x_t, \hat{s}_{t:t+H-1}; \thetacontrol)$, which implicitly depends on the encoder/decoder parameters $\thetaencoder,\thetadecoder$ via the forecast $\hat{s}$.

\tbf{Control Cost:}
Our main objective is to minimize end-to-end control cost $\Jcontrol$, 
which depends on initial state $x_0$ and controls $\hat{u}_{0:T-1}$, which in turn depend on the \textit{forecast} $\shat_{0:T-1}$. For a simpler notation, we use bold variables to define the full timeseries, i.e., $\boldu := u_{0:T-1}$, $\bolds := s_{0:T-1}$, $\bolduhat := \hat{u}_{0:T-1}$ and $\boldshat := \shat_{0:T-1}$. The control cost $\Jcontrol$ is a sum of stage costs $c(x_t, \hat{u}_t)$ and terminal cost $c_T(x_T)$:
$\Jcontrol(\bolduhat; x_0, \bolds) = c_T(x_T) + \sum_{t=0}^{T-1} c(x_t, \hat{u}_t)$, where
$x_{t+1} = f(x_t, \hat{u}_t, s_t), t \in \{0, \cdots, T-1\}$.    
Importantly, the above plant dynamics $f$ evolve according to true timeseries $s_t$, but controls $\hat{u}_t$ are enacted with possibly noisy forecasts $\shat_t$. 

\tbf{Forecasting Errors:}
In practice, a designer often wants to visualize decoded forecasts $\shat$ to debug anomalies or view trends. While our principal goal is to minimize the control errors and cost associated with forecast $\shat$, we allow a designer to \textit{optionally} penalize mean squared prediction error (MSE). This penalty incentivizes a forecast $\shat_t$ to estimate the key trends of $s_t$, serving as a regularization term: 
$\Jforecast(\bolds, \boldshat) = \frac{1}{T} \sum_{t=0}^{T-1} \vert \vert s_t - \shat_t \vert \vert^2_2.$

\tbf{Overall Weighted Cost:}
Given our principal objective of minimizing control cost and optionally penalizing prediction error, we combine the two costs using a user-specified weight $\lambdaforecast$. 
Importantly, we try to minimize the \textit{additional} control cost $\Jcontrol(\bolduhat; x_0, \bolds)$ incurred by using forecast $\boldshat$ instead of true timeseries $\bolds$, yielding overall cost:
\begin{align}
     \Joverall(\boldu, \bolduhat, \bolds, \boldshat; x_0, \lambdaforecast) = \frac{1}{T}\big(\underbrace{\Jcontrol(\bolduhat; x_0, \bolds) - \Jcontrol(\boldu; x_0, \bolds)}_{\mathrm{extra ~control~ cost}}\big) + \lambdaforecast \Jforecast(\bolds, \boldshat). 
    \label{eq:weighted_cost}
\end{align}
The total cost implicitly depends on controller, encoder, and decoder parameters via controls $\boldu$ and $\bolduhat$ and the forecast $\boldshat$.
Having defined the encoder/decoder and controller, we now formally define the problem addressed in this paper.
\begin{problem}[Data Compression for Cooperative Networked Control]
\label{prob:codesign}
We are given a controller $\pi(; \thetacontrol)$ with fixed, pre-trained parameters $\thetacontrol$, fixed bottleneck dimension $\zbottleneck$, and perfect measurements of internal controller state $x_{0:T}$. Given a true exogenous timeseries $s_{0:T-1}$ drawn from data distribution $\mathcal{D}$, find encoder and decoder parameters $\thetaencoder, \thetadecoder$ to minimize the weighted control and forecasting cost (Eq. \ref{eq:weighted_cost}) with weight $\lambdaforecast$:
\begin{align*}
    \thetaencoder^{*}, \thetadecoder^{*} & = \argmin_{\thetaencoder, \thetadecoder} \quad \expec_{s_{0:T-1} \sim \mathcal{D}} [\Joverall(\boldu, \bolduhat, \bolds, \boldshat; x_0, \lambdaforecast)], ~\mathrm{where~} \\ 
    & \phi_t = g_{\mathrm{encode}}(s_{t-W+1:t}; \thetaencoder), ~\phi_t \in \reals^{\zbottleneck} \\
    & \shat_{t:t+H-1} = g_{\mathrm{decode}}(\phi_t; \thetadecoder), \\
    & \hat{u}_t = \pi(x_t, \shat_{t:t+H-1}; \thetacontrol), ~u_t = \pi(x_t, s_{t:t+H-1}; \thetacontrol), \\ 
    & x_{t+1} = f\big(x_t, \hat{u}_t, s_t), \quad \mathrm{and} \quad x_t \in \mathcal{X}, \hat{u}_t \in \mathcal{U}, \quad t \in \{0, \cdots, T-1\}.
\end{align*}
% \begin{align*}
%     \thetaencoder^{*}, \thetadecoder^{*} & = \argmin_{\thetaencoder, \thetadecoder} \expec_{s_{0:T-1} \sim \mathcal{D}} \Joverall(\boldu, \bolduhat, \bolds, \boldshat; x_0, \lambdaforecast), ~\mathrm{where~} \\ 
%     & \phi_t = g_{\mathrm{encode}}(s_{t-W+1:t}; \thetaencoder), ~\phi_t \in \reals^{\zbottleneck} \\
%     & \shat_{t:t+H-1} = g_{\mathrm{decode}}(\phi_t; \thetadecoder), \\
%     & \hat{u}_t = \pi(x_t, \shat_{t:t+H-1}; \thetacontrol), ~u_t = \pi(x_t, s_{t:t+H-1}; \thetacontrol), \\ 
%     & x_{t+1} = f\big(x_t, \hat{u}_t, s_t), \quad \mathrm{and} \quad x_t \in \mathcal{X}, \hat{u}_t \in \mathcal{U}, \quad t \in \{0, \cdots, T-1\}.
% \end{align*}
\end{problem}

\tu{Technical Novelty and Practicality of our Co-design Problem:} 

Having formalized our problem, we can now articulate how it differs from classical networked control and tele-operation \cite{hespanha2007survey,borkar1997lqg,tatikonda2004control,tatikonda2004stochastic,kostina2019rate}, compressed sensing \cite{donoho2006compressed,eldar2012compressed}, and certainty-equivalent control \cite{van1981certainty,mania2019certainty}.
First, we can not readily apply the classical separation principle \cite{wonham1968separation} of Linear Quadratic Gaussian (LQG) control, which proscribes how to \textit{independently} design a timeseries estimator, such as the Kalman Filter \cite{kalman}, and a ``certainty-equivalent'' controller (the linear quadratic regulator) for optimal performance. This is because the timeseries owner measures a \textbf{non-stationary timeseries} $s_t$ (e.g. spatiotemporal cell demand patterns), without an analytical process model for standard Kalman Filtering, motivating our subsequent use of learned DNN forecasters. Second, due to \textbf{data-rate constraints}, we must prioritize task-relevant features as opposed to equally weighting and sending the full $\hat{s}_t$, which a classic state observer in LQG would do. 

Moreover, even when the estimator and controller are separated by a bandwidth-limited network and the separation principle does not hold \cite{6146405}, our setting still differs from classical networked control \cite{hespanha2007survey,borkar1997lqg,tatikonda2004control,tatikonda2004stochastic,kostina2019rate}. These works assume that both the full plant state $x_t$ and controls $u_t$ are encoded and transmitted between a remote controller and plant. In stark contrast, the only transmitted data in our setting is \textit{external} information $s_t$ from a network operator, which can improve an independent controller's \textit{local} decisions $u_t$ based on its internal state $x_t$. As such, simply grouping controller state $x_t$ and network timeseries $s_t$ into a joint state for classical tele-operation is infeasible, since $x_t$ and $s_t$ are measured at different locations by different entities. 
%Our novel setting is important in smart cities to limit data sharing between IoT sensors, network operators, and controllers. 
In essence, Prob. \ref{prob:codesign} formalizes how a network operator can provide significant value to an independent controller by judicious data sharing.

%we capture many practical scenarios where a controller measures its internal state $x_t$ and makes decisions $u_t$ locally. Crucially, we model how \textit{external} information $s_t$ from a network operator can improve a controller's local decision-making, which is the only transmitted data. Such scenarios are important in smart cities to limit data sharing between IoT sensors, network operators, and controllers. 
%
%In contrast, we capture many practical scenarios where a controller measures its internal state $x_t$ and makes decisions $u_t$ locally. Crucially, we model how \textit{external} information $s_t$ from a network operator can improve a controller's local decision-making, which is the only transmitted data. 
%Such scenarios are important in smart cities to limit data sharing between IoT sensors, network operators, and controllers. 

%In contrast, we model how \textit{external} information $s_t$ from a network operator can significnatly improve an independent controller's local decisions $u_t$ coupled with full measurements of its internal state $x_t$.
%Crucially, we model how \textit{external} information $s_t$ from a network operator can improve a controller's local decision-making, which is the only transmitted data. 
%Importantly, forecast $s_t$ is the only transmitted data, which is cruical to limit data sharing between IoT sensors, network operators, and controllers in smart cities. 
%\SC{Finally, simply grouping controller state $x_t$ and timeseries $s_t$ into a joint state and applying classical tele-operation is infeasible because $x_t$ and $s_t$ are measured at different locations by distinct entities}.

%In essence, Prob. \ref{prob:codesign} formalizes how a network operator can provide significant value to an independent controller by measuring and forecasting task-relevant trends of $s_t$.}
%Third, the operator only sees \textit{uncontrollable} $s_t$, while the application  measures \textit{controllable}, internal state $x_t$. Thus, augmenting them into a \textit{fully-controllable} macro-state $\bar{x}_t = [x_t, s_t]$ like classical networked LQR is infeasible due to the presence of \textbf{distinct information spaces} and inappropriate since $s_t$ is not a controlled state. 
%Crucially, we model how a network operator can provide significant value to an independent controller by sharing task-relevant trends of $s_t$ to improve decision-making, which is the only transmitted data. Such scenarios are important in smart cities to limit data sharing between IoT sensors, network operators, and controllers.
%In contrast, we capture many practical scenarios where a controller measures its internal state $x_t$ and makes decisions $u_t$ locally. 
%Crucially, we model how \textit{external} information $s_t$ improves decisions, which is the only transmitted data. Such scenarios are important in smart cities to limit data sharing between IoT sensors, network operators, and controllers.


\section{Forecaster and Controller Co-design}
\label{sec:algorithm}
Prob. \ref{prob:codesign} is of wide scope, and can encompass both neural network forecasters and controllers. For intuition, we first provide analytical results for an \textit{input-driven} LQR problem in Sec. \ref{subsec:input_driven_LQR}. We then use such insights in a general learning algorithm that scales to DNN forecasters in Sec. \ref{subsec:alg_codesign}.

\subsection{Input-Driven Linear Quadratic Regulator (LQR)}
\label{subsec:input_driven_LQR}

We first consider a simple instantiation of Prob. \ref{prob:codesign} with 
linear dynamics\footnote{\JC{Transition noise is not added here due to certainty equivalence of input-driven LQR, as shown in Appendix Sec. \ref{subsec:lqr_transition_noise}.}}, no state or control constraints, and a quadratic control
cost. Since the dynamics have linear dependence on the exogenous input $s$, we
refer to this setting as an \textit{input-driven} LQR problem. We first analyze the problem
when controls are computed for the full-horizon from $t=0$ to $T=H$ and then extend
to receding-horizon control (MPC) in Sec. \ref{subsec:alg_codesign}. The dynamics and control cost are:
\begin{align}
    & x_{t+1} = A x_t + B u_t + C s_t, \\ 
    & \JMPC = \sum_{t=0}^{H} x_t^\top Q x_t + \sum_{t=0}^{H-1} u_t^\top R u_t, 
    \label{eq:input_driven_LQR} 
\end{align}
where $Q, R$ are positive definite.
Our first step is to determine the optimal control.
% To formally define Prob. \ref{prob:codesign} for input-driven LQR, we first find an optimal controller $\pi$ (Eq. \ref{eq:control_policy}).
Given the linear dynamics, for all times $i \in \{0, \cdots, H-1\}$, each $x_{i+1}$ is a linear function of initial condition $x_0$ and the \textit{full future} control vector $\boldu$ and $\bolds$:
\begin{align}
x_{i+1} = A^{i+1} x_0 + \bm{M_{i}} \boldu + \bm{N_{i}} \bolds, ~~\text{where}~~\label{eq:rewrite_x_t}
\end{align}
\begin{align*}
\bm{M_{i}} = \begin{bmatrix}
A^{i}B & A^{i-1}B & \cdots & B & \bm{0}
\end{bmatrix} \in \mathbb{R}^{n \times mH}, \bm{N_{i}} = \begin{bmatrix}
    A^{i}C & A^{i-1}C & \cdots & C & \bm{0}
\end{bmatrix} \in \mathbb{R}^{n \times pH}.
\end{align*}

Therefore, given $x_0$ and vector $\bolds$, control cost $\JMPC$ is a quadratic function of $\boldu$:
\begin{align}
\JMPC(\boldu; x_0, \bolds) ~=~ &  \boldu^\top (\underbrace{\bm{R} + \sum_{i=0}^{H-1} \bm{M_{i}}^\top Q \bm{M_{i}}}_{\bm{K}}) \boldu + 2[\underbrace{\sum_{i=0}^{H-1} \bm{M_{i}}^\top Q (A^{i+1} x_0 + \bm{N_{i}} \bolds)}_{\bm{k}(x_0, \bolds)}]^\top \boldu +~ \text{constant}, \label{eq:quadratic_jmpc}
\end{align}
where the constant of $\sum_{i=0}^{H-1} (A^{i+1} x_0 + \bm{N_{i}} \bolds)^\top Q (A^{i+1} x_0 + \bm{N_{i}} \bolds)$ is independent of $\boldu$, and $\bm{R} =\text{blockdiag}(R, \cdots, R) \in \mathbb{R}^{mH \times mH}$. Clearly, $\bm{K}$ is positive definite and $\JMPC$ is strictly convex.
Given the convex quadratic cost, the optimal control is $\bolduMPCopt$, where $\bolduMPCopt = - \bm{K}^{-1} \bm{k}(x_0, \bolds)$. However, given a possibly noisy forecast $\boldshat$, we would instead plan and enact controls denoted by $\bolduMPChat$, where $\bolduMPChat = - \bm{K}^{-1} \bm{k}(x_0, \boldshat)$. Thus, the sensitivity of such controls to forecast errors is: 
\begin{align}
& \bolduMPChat - \bolduMPCopt = - \bm{K}^{-1} (\bm{k}(x_0, \boldshat) - \bm{k}(x_0, \bolds))
= - \bm{K}^{-1} \underbrace{(\sum_{i=0}^{H-1} \bm{M_{i}}^\top Q \bm{N_{i}})}_{\bm{L}} (\boldshat - \bolds),
\label{eq:control_error}
\end{align}
and the sensitivity of the control cost to forecast errors is:
\begin{align}
    \JMPC(\bolduMPChat; x_0, \bolds) - & \JMPC(\bolduMPCopt; x_0, \bolds)
= (\bolduMPChat - \bolduMPCopt)^\top \bm{K} (\bolduMPChat - \bolduMPCopt) = (\boldshat - \bolds)^\top \underbrace{\bm{L}^\top \bm{K}^{-1} \bm{L}}_{\text{co-design matrix} ~\Psi} (\boldshat - \bolds), 
\label{eq:codesign_intuitive}
\end{align}
where we term the positive semi-definite \textit{co-design matrix} $\Psi = \bm{L}^\top \bm{K}^{-1} \bm{L}$.
We now combine the extra control cost and prediction error to calculate the total cost as: 
\begin{align}
    \Joverall & = \frac{1}{H}\big(\underbrace{(\boldshat - \bolds)^\top\Psi(\boldshat - \bolds)}_{\mathrm{extra~control~cost}} + \lambdaforecast \underbrace{(\boldshat - \bolds)^\top (\boldshat - \bolds)}_{\mathrm{prediction~error}}\big)
    = \frac{1}{H}\big((\boldshat - \bolds)^\top (\Psi + \lambdaforecast I) (\boldshat - \bolds)\big).
\end{align}
The above expression leads to an intuitive understanding of co-design. The co-design matrix $\Psi$ in Eq. \ref{eq:codesign_intuitive} essentially weights the error in elements of $\boldshat$ based on their importance to the ultimate control cost. Thus, our approach is fundamentally \textit{task-aware} since the co-design matrix depends on LQR's dynamics, control, and cost matrices as shown in Eq. \ref{eq:control_error} and \ref{eq:codesign_intuitive}. The optional weighting of prediction error with $\lambdaforecast$ acts as a regularization term. Moreover, we now show that we can reduce input-driven LQR to a low-rank approximation problem, which 
allows us to find an \textit{analytic} expression for an optimal encoder/decoder. 
%that solve Prob. \ref{prob:codesign}}.

%\subsection{Input-Driven LQR is Low-Rank Approximation}
\textbf{Input-Driven LQR is Low-Rank Approximation.}
%\label{subsec:low_rank_approximation}
Given the above expressions for the total cost, we now assume a simple parametric
model for the encoder and decoder to formally write Prob. \ref{prob:codesign} for the toy input-driven LQR setting. Specifically, we assume a linear encoder $E \in \reals^{\zbottleneck \times \pH}$ maps true exogenous input $\bolds$ to representation $\phi = E\bolds$, where $\phi \in \reals^{\zbottleneck}$. 
Then, linear decoder matrix $D \in \reals^{\pH \times \zbottleneck}$ yields decoded timeseries $\boldshat = D\phi = DE\bolds$. In practice, we often have a training dataset consisting of $N$ samples of exogenous input $\bolds$ drawn from a data distribution $\bolds \sim \mathcal{D}$. These samples can be arranged as columns in a matrix $\boldS \in \reals^{\pH \times N}$. To learn an encoder $E$ and decoder $D$ from $N$ samples $\boldS$ at once, we can express our problem as:
\begin{align}
    \argmin_{D,E} \quad & \sum_{i=1}^{N}(\boldShat_i - \boldS_i)^\top (\Psi + \lambdaforecast I) (\boldShat_i - \boldS_i), \quad{\text{~where~}} \notag \\
    & \boldShat = DE \boldS, ~\text{rank}(D) \le \zbottleneck \text{~and~rank}(E) \le \zbottleneck, 
    \label{eq:low_rank_approximation}
\end{align}
where $\boldS_i$ and $\boldShat_i$ represent the $i$-th column vector of $\boldS$ and $\boldShat$. We now characterize the input-driven LQR problem.

\begin{prop}[Linear Weighted Compression]
    Input-driven LQR (Eq. \ref{eq:low_rank_approximation}) is a low-rank approximation problem, which admits an analytical solution for an optimal encoder and decoder pair $(E, D)$. \label{prop:input_dirven_lqr}
\end{prop}
\begin{proof}
    \renewcommand{\qedsymbol}{}
We first re-write the objective of the input-driven LQR problem (Eq. \ref{eq:low_rank_approximation}) as: 
$\sum_{i=1}^{N}(\boldShat_i - \boldS_i)^\top (Y \Lambda Y^\top) (\boldShat_i - \boldS_i)
= \doublevert \Lambda^{\frac{1}{2}} Y^\top \boldShat - \Lambda^{\frac{1}{2}}Y^\top \boldS \doublevert_F^2$, 
where $Y \Lambda Y^\top$ is the eigen-decomposition of the positive definite matrix $\Psi + \lambdaforecast I$ and $\doublevert . \doublevert_F$ represents the Frobenius norm of a matrix. Thus, the problem can be written as: 
    \begin{align}
        \argmin_{D,E} \quad \doublevert \underbrace{\Lambda^{\frac{1}{2}}Y^\top DE \boldS}_{\mathrm{approximation}} - \underbrace{\Lambda^{\frac{1}{2}}Y^\top \boldS}_{\mathrm{original}} \doublevert_F^2, \quad{\text{~where~}} ~\text{rank}(D) \le \zbottleneck \text{~and~rank}(E) \le \zbottleneck, 
        \label{eq:low_rank_approximation_final}
    \end{align}
    which is the canonical form of a low-rank approximation problem. By the Eckhart-Young theorem, the solution to the input-driven LQR problem (Eq. \ref{eq:low_rank_approximation_final}) is the rank $Z$ truncated singular value decomposition (SVD) of original matrix $\Lambda^{\frac{1}{2}}Y^\top\bold{S}$, denoted by $U \Sigma V^\top$. In the truncated SVD, $U\in\reals^{\pH \times Z}$ is semi-orthogonal, $\Sigma \in \reals^{Z\times Z}$ is a diagonal matrix of singular values, and $V \in \reals^{N \times Z}$ is semi-orthogonal. Further, an encoder of $E = U^\top \Lambda^{\frac{1}{2}} Y^\top$ and decoder of $D = (\Lambda^{\frac{1}{2}} Y^\top)^{-1}U$ solve the problem since:
\begin{align*}
    \underbrace{\Lambda^{\frac{1}{2}}Y^\top DE \bold{S}}_{\mathrm{approximation}}
    = & ~\Lambda^{\frac{1}{2}}Y^\top \underbrace{(\Lambda^{\frac{1}{2}}Y^\top)^{-1}U}_{D} \underbrace{U^\top \Lambda^{\frac{1}{2}}Y^\top}_{E} \bold{S}
    = ~U (U^\top \Lambda^{\frac{1}{2}} Y^\top \bold{S} ) = \underbrace{U \Sigma V^\top}_{\mathrm{optimal~rank~Z~approximation}}.  
\end{align*}
%Due to space limits, further details are in the supplement. 
%In the last step, we used the fact that $U^\top Y^\top \Lambda^{\frac{1}{2}}\bold{S} = U^\top (U \Sigma V^\top + U' \Sigma' V'^\top) = \Sigma V^\top$ because $U$ and $U'$ are orthogonal.
\end{proof}

A similar analysis for a linear encoder-decoder structure for networked inference, not control, is presented in \cite{DBLP:conf/rss/NakanoyaCADKP21}. The key difference from our current paper is our problem setup is for control, not networked inference. \JC{Moreover, our result differs from existing LQR literatures with exogenous input, such as \cite{singh2017extended}, since our exogenous input $s_t$ is subject to to a network bottleneck and encoder/decoder, which is the crux of our Prob. \ref{prob:codesign}; and our total cost includes the extra control cost due to mis-estimation of $s_t$, rather than simply the prediction error of $s_t$.}

\textbf{Compression benefits: }
Casting input-driven LQR as low-rank approximation provides significant intuition.
As shown in Proposition \ref{prop:input_dirven_lqr}, the optimal encoder/decoder depend on the truncated SVD of $\Lambda^{\frac{1}{2}} Y^\top \boldS$, which takes into account the control task via the co-design matrix, importance of prediction errors via $\lambdaforecast$, and statistics of the input via $\boldS$. \SC{We achieved strong compression benefits for simulations of input-driven LQR (provided in supplement Fig. \ref{fig:main_pca_full} due to space limits).} 

%%%%%%%%%%%%%%%%%%%%%%%%%%%%%%%%%%%%%%%%%%%%
%\subsection{Transitioning to Model Predictive Control (MPC)}
\textbf{Transitioning to Model Predictive Control (MPC).}
In practice, we often have forecasts for a short horizon $H < T$. Then, starting from any state $x_t$, MPC will plan a sequence of controls $\hat{u}_{t:t+H-1}$, enact the first control $\hat{u}_t$, and then re-plan with the next forecast. If we replace the horizon to $H < T$ in the input-driven LQR analysis in Sec. \ref{subsec:input_driven_LQR}, $\bolduMPCopt = - \bm{K}^{-1} \bm{k}(x_0, \bolds)$ gives the optimal control for a \textit{short-horizon} $H$ and we can encode/decode using a low rank approximation as in Prop. 1. While the performance is \textit{not} necessarily optimal for the full duration $T$, MPC performs extremely well in practice, yielding even better compression gains, as shown in the supplement (Fig. \ref{fig:main_pca}). 

%\begin{figure*}[ht]
%\vskip 0.2in
\begin{wrapfigure}{L}{0.5\textwidth}
\begin{minipage}{0.5\textwidth}
\begin{algorithm}[H]
   \caption{Compression Co-design for Control}
   \label{alg:codesign}
   \begin{algorithmic}[1]
       \STATE Set forecast weight $\lambdaforecast$, bottleneck size $\zbottleneck$ \label{alg:lambda}
       \STATE Initialize encoder/decoder parameters $\thetaencoder^{0}$, $\thetadecoder^{0}$ randomly, and fix controller parameters $\thetacontrol$ \label{alg:init_controller}
       \FOR{$\tau~ \gets 0$ \KwTo $\Nepochs-1$}{
               \STATE Initialize Controller State $x_0 \in \mathcal{X}$ 
           \FOR{$t~ \gets 0$ \KwTo $T-1$}{
               \STATE Encode $\phi_t = g_{\mathrm{encode}}(s_{t-W+1:t}; \thetaencoder^{\tau})$ \label{alg:encode}
               \STATE Decode $\hat{s}_{t:t+H-1} = g_{\mathrm{decode}}(\phi_t; \thetadecoder^{\tau})$ \label{alg:decode}
               \STATE Enact $\hat{u}_{t} = \pi(x_t, \hat{s}_{t:t+H-1}; \thetacontrol)$ 
               \STATE Propagate  $x_{t+1} \gets f(x_t, \hat{u}_t, s_t)$ 
               \STATE $u_{t} = \pi(x_t, s_{t:t+H-1}; \thetacontrol)$ (For Training Only) \label{alg:plan}
            }
           \ENDFOR
           \STATE $\thetaencoder^{\tau + 1}, \thetadecoder^{\tau + 1} \gets$\par$\textsc{BackProp} \big[\Joverall(\boldu,\bolduhat,\bolds,\boldshat; x_0, \lambdaforecast)\big]$ \label{alg:backprop}
       }
       \ENDFOR
       \STATE Return learned parameters $\thetaencoder^{\Nepochs}, \thetadecoder^{\Nepochs}$
    \end{algorithmic}
\end{algorithm}

\end{minipage}
\end{wrapfigure}
%\vskip -0.2in
%\vskip -0.2in
%\end{figure*}


We also note a practitioner can adopt a simple cost function based on MPC that complements Eq. \ref{eq:weighted_cost}. The MPC controller $\pi$ will optimize the cost $\Joverall$ given a short-horizon forecast $\shat_{t:t+H-1}$, but only enact the \textit{first} control $\hat{u}_t = \pi(x_t, \shat_{t:t+H-1}; \thetacontrol)$. Meanwhile, the best first control MPC can take is $u_t = \pi(x_t, s_{t:t+H-1}; \thetacontrol)$ with perfect knowledge of $s$ for horizon $H$. Thus, our insight is that we can penalize the errors in \textit{enacted} controls $\hat{u}_t$ during training and regularize for prediction error, using cost:
$\frac{1}{T}\big(\sum_{t=0}^{T-1} \doublevert \hat{u}_t - u_t \doublevert_2^2 + \lambdaforecast \doublevert \hat{s}_{t} - s_t \doublevert_2^2\big)$.
In our experiments, we observed strong performance by optimizing for the cost Eq. \ref{eq:weighted_cost}, as well as the above cost, which optimizes $\Joverall$ over a short-horizon for MPC.
We now crystallize these insights from input-driven LQR into a formal algorithm that applies to data-driven MPC.
%\JC
%{
%However, for MPC in general we are not able to relate control difference directly to forecasting error (like what we did in Eq. \ref{eq:control_error}), because the $x_t$ evolves differently with optimal and non-optimal control and hence also contributes to control difference. Therefore,  
%we replace the extra control cost in Eq. \ref{eq:weighted_cost} with the average control error, defined as:
%\begin{small}
%\begin{align*}
%    \frac{1}{T} \sum_{t=0}^{T-1} \doublevert \pi(x_t, s_{t:t+H-1}; \thetacontrol) - \pi(x_t, s_{t:t+H-1}; \thetacontrol) \doublevert_2^2
%\end{align*}
%\end{small}
%where $\pi(x_t, s_{t:t+H-1}; \thetacontrol)$ and $\pi(x_t, s_{t:t+H-1}; \thetacontrol)$ adopt the same $x_t$. Our ideology is, the controller with a small average control error in general has a similar evolution of $x_t$ as the optimal controller, which further produces near-optimal control cost. 
%
%(Can remove the definition in Sec. 5.2 after adding this)  
%}
%%%%%%%%%%%%%%%%%%%%%%%%%%%%%%%%%%%%%%%%%%%%
\subsection{Algorithm to Co-design Forecaster and Controller}
\label{subsec:alg_codesign}
For more complex scenarios than LQR, it is challenging to provide analytical forms of an optimal encoder and decoder. Thus, we present a heuristic algorithm to solve Prob. \ref{prob:codesign} in Algorithm \ref{alg:codesign}. 
Our key technical insight is that, if the encoder, decoder, and controller are differentiable, we can write:
\begin{align}
    \frac{\nabla \Joverall(\boldu, \bolduhat, \bolds, \boldshat; x_0, \lambdaforecast)}{\nabla \thetaencoder} = \frac{\nabla \Joverall(\boldu, \bolduhat, \bolds, \boldshat; x_0, \lambdaforecast)}{\nabla (\boldshat - \bolds)} \times \frac{\nabla (\boldshat- \bolds)}{\nabla \thetaencoder},
\end{align}
and likewise for $\thetadecoder$. The first term captures the sensitivity of the control cost with respect to prediction errors and the second propagates that sensitivity to the forecasting model.
Crucially, the gradient of $\Joverall$ can be obtained from recent methods that learn differentiable MPC controllers \cite{agrawal2020learningcontrol,amos2018differentiable}. 

% %\begin{figure*}[ht]
%\vskip 0.2in
\begin{wrapfigure}{L}{0.5\textwidth}
\begin{minipage}{0.5\textwidth}
\begin{algorithm}[H]
   \caption{Compression Co-design for Control}
   \label{alg:codesign}
   \begin{algorithmic}[1]
       \STATE Set forecast weight $\lambdaforecast$, bottleneck size $\zbottleneck$ \label{alg:lambda}
       \STATE Initialize encoder/decoder parameters $\thetaencoder^{0}$, $\thetadecoder^{0}$ randomly, and fix controller parameters $\thetacontrol$ \label{alg:init_controller}
       \FOR{$\tau~ \gets 0$ \KwTo $\Nepochs-1$}{
               \STATE Initialize Controller State $x_0 \in \mathcal{X}$ 
           \FOR{$t~ \gets 0$ \KwTo $T-1$}{
               \STATE Encode $\phi_t = g_{\mathrm{encode}}(s_{t-W+1:t}; \thetaencoder^{\tau})$ \label{alg:encode}
               \STATE Decode $\hat{s}_{t:t+H-1} = g_{\mathrm{decode}}(\phi_t; \thetadecoder^{\tau})$ \label{alg:decode}
               \STATE Enact $\hat{u}_{t} = \pi(x_t, \hat{s}_{t:t+H-1}; \thetacontrol)$ 
               \STATE Propagate  $x_{t+1} \gets f(x_t, \hat{u}_t, s_t)$ 
               \STATE $u_{t} = \pi(x_t, s_{t:t+H-1}; \thetacontrol)$ (For Training Only) \label{alg:plan}
            }
           \ENDFOR
           \STATE $\thetaencoder^{\tau + 1}, \thetadecoder^{\tau + 1} \gets$\par$\textsc{BackProp} \big[\Joverall(\boldu,\bolduhat,\bolds,\boldshat; x_0, \lambdaforecast)\big]$ \label{alg:backprop}
       }
       \ENDFOR
       \STATE Return learned parameters $\thetaencoder^{\Nepochs}, \thetadecoder^{\Nepochs}$
    \end{algorithmic}
\end{algorithm}

\end{minipage}
\end{wrapfigure}
%\vskip -0.2in
%\vskip -0.2in
%\end{figure*}


In lines 1-2 of Alg.\ref{alg:codesign}, we randomly initialize the encoder and decoder parameters and set the latent representation size $\zbottleneck$ to limit the communication data-rate. Then, we enact control policy rollouts in lines 3-11 for $\Nepochs$ training epochs, each of duration $T$. We first encode and decode the forecast $\boldshat$ (lines 6-7) and pass them to the downstream controller with fixed parameters $\thetacontrol$ (lines 8-10). During training, we calculate the loss by comparing the optimal weighted cost with \textit{true} input $\bolds$ and the forecast $\boldshat$. In turn, this loss is used to train the differentiable encoder and decoder through backpropagation in line 12. Finally, the learned encoder and decoder (line 14) are deployed.

\textbf{Co-design Algorithm Discussion: }
A few comments are in order. First, true input $\bolds$ is only needed during \textit{training}, which is accomplished at a single server using historical data to avoid passing large gradients over a real network. Then, we can periodically re-train the encoder/decoder during online deployment.
Second, our approach also applies when $\thetacontrol$ are parameters of a deep reinforcement learning (RL) policy. 
However, since the networked systems we consider have well-defined dynamical models, we focus our evaluation on model-based control. 
%learning forecasters for model-based control.


\section{Application Scenarios}
\label{sec:scenarios}
We now describe three diverse application scenarios addressed in our evaluation. The scenarios
are linear MPC problems with box control constraints:
\begin{align}
    & x_{t+1} = x_{t} + u_t - s_t, ~~\text{(Dynamics)}  \quad{\text{~where~}} ~~
    u_{\text{min}} \leq u_t \leq u_{\text{max}}. ~~\text{(Constraints)} \label{eq:dynamics_MPC}
\end{align}
Our scenarios have the same state and control dimensions $m=n$, and dynamics/control matrices $A=B=I_{n \times n}$ indicate uniform coupling between controls and the next state. Finally, we have actuation limits $u_{\text{min}}$ and $u_{\text{max}}$. The cost function incentivizes regulation of the state $x_t$ to a set-point $L$. In practice, we often want to penalize states below the set-point, such as inventory shortages where $x_t < L$, more heavily than those above, such as excesses. In the following cost, weights $\gamma_e, \gamma_s, \gamma_u \in \reals^{+}$ govern excesses, shortages, and controls $u_t$ respectively:
\begin{align}
    \Jcontrol(\boldx, \boldu) = \sum_{t=0}^{T} (\gamma_e \doublevert [x_t - L]_{+} \doublevert^2_2 + \gamma_s \doublevert [L - x_t]_{+} \doublevert^2_2)  + \sum_{t=0}^{T-1} \gamma_u \doublevert u_t \doublevert^2_2,
\label{eq:biased_LQR_cost}
\end{align}
where $[x]_{+}$ represents the positive elements of a vector. We focus on linear MPC with box constraints and a flexible quadratic cost (Eq. \ref{eq:biased_LQR_cost}) since it is a canonical problem \cite{Camacho2013,borrelli2017predictive} with wide applications in networked systems. \SC{However, to show the generality of co-design, we provide strong experimental results for a mobile video streaming application with \textit{noisy, non-linear} dynamics in 
% Appendix 
Sec. \ref{subsec:nonlinear}}.
We evaluate diverse MPC settings coupled with an array of neural network forecasters.
%Our diverse scenarios couple linear MPC with neural-network forecasting models.
%Linear MPC with box constraints and a flexible quadratic cost (Eq. \ref{eq:biased_LQR_cost}) is a canonical problem with wide engineering applications \cite{Camacho2013,borrelli2017predictive}.
%%\subsection{Data-Center Temperature Control}
%\textbf{Data-Center Temperature Control:}
%Consider an idealized datacenter cooling problem, where $x \in \statedim$ represents the temperature of $n$ server racks we wish to regulate to a temperature set-point of $L$. External heat disturbances $s \in \exoinputdim$ add to temperature state $x_t$ uniformly with $\beta = 1$ in the dynamics (Eq. \ref{eq:dynamics_MPC}). Disturbances are measured by $p=n$ IoT temperature sensors, such as from nearby heating units in the building. Our objective is to select control inputs $u \in \controldim$ to cool the server racks \textit{anticipating} heat disturbances $s$ from the $p$ IoT sensors. The cost function (Eq. \ref{eq:biased_LQR_cost}) has $\gamma_e = \gamma_s = \gamma_u = 1$ to equally penalize temperature deviation and cooling effort. Finally, we captured stochastic timeseries of temperature, pressure, humidity, and light from the Google Edge Tensor Processing Unit (TPU)'s environmental sensor board. Further details of our real-world dataset are provided in the supplement. 

\textbf{Smart Factory Regulation with IoT Sensors:}
We consider an \textit{idealized} scenario similar to datacenter temperature control \cite{recht2019tour}, where $x_t \in \statedim$ represents the temperature, humidity, pressure and light for $\frac{n}{4}$ machines in a smart factory, each of whose 4 sensor measurements we want to regulate to a set-point of $L$. External heat, humidity, and pressure disturbances $s \in \exoinputdim$ add to state $x_t$ in the dynamics (Eq. \ref{eq:dynamics_MPC}). Disturbances are measured by $p=n$ IoT sensors, such as from nearby heating units. Our objective is to select control inputs $u \in \controldim$ to regulate the environment \textit{anticipating} disturbances $s$ from the $p$ IoT sensors. The cost function (Eq. \ref{eq:biased_LQR_cost}) has $\gamma_e = \gamma_s = \gamma_u = 1$ to equally penalize deviation from the set-point and regulation effort. Finally, we collected two weeks of stochastic timeseries of temperature, pressure, humidity, and light from the Google Edge Tensor Processing Unit (TPU)'s environmental sensor board for our experiments, as detailed in the supplement.
% $x_t \in \statedim$ represents the temperature, humidity, and pressure of $n$ machines in a smart factory we wish to regulate to a set-point of $L$ per machine. External heat, humidity, and pressure disturbances $s \in \exoinputdim$ add to state $x_t$ in the dynamics (Eq. \ref{eq:dynamics_MPC}). Disturbances are measured by $p=n$ IoT temperature sensors, such as from nearby heating units. Our objective is to select control inputs $u \in \controldim$ to cool and regulate the machine pressure \textit{anticipating} disturbances $s$ from the $p$ IoT sensors. The cost function (Eq. \ref{eq:biased_LQR_cost}) has $\gamma_e = \gamma_s = \gamma_u = 1$ to equally penalize deviation from the set-point and regulation effort. Finally, we collected two weeks of stochastic timeseries of temperature, pressure, humidity, and light from the Google Edge Tensor Processing Unit (TPU)'s environmental sensor board for our experiments, as detailed in the supplement.  

%\subsection{Taxi Dispatch Based on Cell Demand Data}
\textbf{Taxi Dispatch Based on Cell Demand Data: }
In this scenario, state $x_t \in \statedim$ represents the difference between the number of free taxis and waiting passengers at $n$ city sites, so $x_t > 0$ represents idling taxis while $x_t < 0$ represents queued passengers. Control $u_t \in \controldim$ represents how many taxis are dispatched to serve queued passengers. Exogenous input $s_t \in \exoinputdim$ represents how many new passengers join the queue at time $t$. Of course, the taxi service has a historical forecast of $s_t$, but the cellular operator can use city-wide mobility data to \textit{improve} the forecast. 
Our goal is to regulate $x_t$ to $L=0$ to neither have waiting passengers nor idling taxis. In the cost function (Eq. \ref{eq:biased_LQR_cost}), we have $\gamma_e = 1, \gamma_s = 100$ and $\gamma_u = 1$ to heavily penalize customer waiting time for long queues. Our simulations use 4 weeks of stochastic cell demand data from Melbourne, Australia from \cite{chinchali2018cellular}.

\textbf{Battery Storage Optimization: }
Our final scenario is inspired by a closely-related work to ours \cite{donti2017task}, who consider how a \textit{single} battery must be charged or discharged based on electricity price forecasts. Since our setting involves a vector timeseries $s$, we consider electrical load forecasts from \textit{multiple} markets. Thus, we used electricity demand data from the same PJM operator as in \cite{donti2017task}, but from multiple markets in the eastern USA \cite{PJM}. Specifically, state $x_t \in \statedim$ represents the charge on $\ninput$ batteries and control $u_t \in \controldim$ represents how much to charge the battery to meet demand. Timeseries $s_t \in \exoinputdim$ represents the demand forecast at the locations of the $n$ batteries, where $p=n$. 
In the cost function (Eq. \ref{eq:biased_LQR_cost}), we desire a battery of total capacity $2L$ to reach a set-point where it is half-full, which, as per \cite{donti2017task}, allows flexibly switching between favorable markets. Further, we set $\gamma_e = \gamma_s = \gamma_u = 1$. 
%The box constraints enforce a positive charge between $0$ and $2L$, so $0 \le x_t \le 2L$.
%Further, to ensure demand can be met, we penalize lower charge stores more heavily than excess charge with $\gamma_e \gg \gamma_s$. 

%\JC
%{
%[IoT, PJM] Basic formulation (save us from using $A, B, C$ ...):
%\begin{align*}
%x_{t+1} = x_{t} + u_t - s_t, \\
%J = \sum_{t=0}^{T} ||x_t||^2_2 + \sum_{t=0}^{T-1} ||u_t||^2_2 
%\end{align*}
%
%[Cell] Biased cost function:
%\begin{align*}
%J = \sum_{t=0}^{T} (\gamma_e ||x_{t, e}||^2_2 + \gamma_s ||x_{t, s}||^2_2) + \sum_{t=0}^{T-1} ||u_t||^2_2 
%\end{align*}
%where $x_{t, e}$ and $x_{t, s}$ are the positive and negative part of $x_t$, respectively.
%
%[IoT 2nd Version] Box-constrained dynamics:
%\begin{align*}
%u_{\text{min}} \leq u_t \leq u_{\text{max}} \\
%\end{align*}
%}


\section{Evaluation}
\label{sec:evaluation}
The goal of our evaluation is to demonstrate that our co-design
algorithm achieves near-optimal control cost, but for much smaller representations $\zbottleneck$ compared to task-agnostic methods. 

\tbf{Metrics.} We evaluate the following metrics: 
1) We quantify the \textbf{control cost} for various bottleneck sizes $\zbottleneck$, relative to the \textit{optimal cost} when ground-truth input $s$ is shared \textit{without} a network bottleneck.
2) To quantify the benefits of sending a representation of size $\zbottleneck$ compared to the full forecast $\shat_{t:t+H-1}$ of size $\pH$, we define the \textbf{compression gain} as $\frac{\pH}{\zbottleneck}$. We also compare the minimum bottleneck $Z$ required to achieve within $5\%$ of the optimal cost for all benchmarks. 
3) Since the objective of Prob. \ref{prob:codesign} also incorporates prediction error, we quantify the \textbf{MSE forecasting error} for various $\zbottleneck$.

% \begin{figure*}[t]
\vskip 0.2in
\begin{center}
\includegraphics[width=0.8\columnwidth]{figures/main_legend.pdf}
%\subfigure[]{
\subfigure{
{\includegraphics[width=0.31\columnwidth]{figures/iot/iot_cost_bottleneck.pdf}}
\label{fig_iot_cost_bottleneck}
}
%\subfigure[]{
\subfigure{
{\includegraphics[width=0.31\columnwidth]{figures/cell/cell_cost_bottleneck.pdf}}
\label{fig_cell_cost_bottleneck}
}
%\subfigure[]{
\subfigure{
{\includegraphics[width=0.31\columnwidth]{figures/pjm/pjm_cost_bottleneck.pdf}}
\label{fig_pjm_cost_bottleneck}
}
%\subfigure[]{
\subfigure{
{\includegraphics[width=0.31\columnwidth]{figures/iot/control_errors.pdf}}
\label{fig_iot_control_errors}
}
%\subfigure[]{
\subfigure{
{\includegraphics[width=0.31\columnwidth]{figures/cell/control_errors.pdf}}
\label{fig_cell_control_errors}
}
%\subfigure[]{
\subfigure{
{\includegraphics[width=0.31\columnwidth]{figures/pjm/control_errors.pdf}}
\label{fig_pjm_control_errors}
}
%\subfigure[]{
\subfigure{
{\includegraphics[width=0.31\columnwidth]{figures/iot/forecast_errors_time_horizon.pdf}}
\label{fig_iot_forecast_errors_time_horizon}
}
%\subfigure[]{
\subfigure{
{\includegraphics[width=0.31\columnwidth]{figures/cell/forecast_errors_time_horizon.pdf}}
\label{fig_cell_forecast_errors_time_horizon}
}
%\subfigure[]{
\subfigure{
{\includegraphics[width=0.31\columnwidth]{figures/pjm/forecast_errors_time_horizon.pdf}}
\label{fig_pjm_forecast_errors_time_horizon}
}
\caption{\textbf{Real-world dataset results: } From left to right, the columns correspond to smart factory regulation from IoT sensors, taxi dispatching with cell demand, and battery storage optimization. \textbf{(Row 1)} Co-design achieves lower cost $\Jcontrol$ for smaller bottlenecks $\zbottleneck$ compared to task-agnostic methods. \textbf{(Row 2)} We also achieve lower error for each dimension $i$ of the vector control, $u(i)$, plotted for a highly-compressed $Z=3$. \textbf{(Row 3)} Co-design heavily reduces forecasting errors for initial horizons that are especially important for MPC's decision-making.}
%While a task-agnostic approach (blue) distributes forecasting errors roughly evenly across a horizon $H$, our co-design approaches heavily reduce errors for initial horizons that are especially important for MPC's decision-making.}
\label{fig_real}
\end{center}
\vskip -0.2in
\end{figure*}
%Aggregated forecasting error for  each relative time index when $Z=3$, under different policies. (j-l) Forecasting error for each $s(i)$ and each relative time index when $Z=3$, under task-agnostic (top) and weighted (bottom) policy, respectively.
%\caption{Results for real data. Columns from left to right corresponds to smart home regulation, taxi dispatching and battery storage optimization, respectively. (a-c) Control cost $J$ under different bottleneck dimension $Z$ and training policies; (d-f) Control error for each $u(i)$ when $Z=3$ under different training policies; (g-i) Aggregated forecasting error for  each relative time index when $Z=3$, under different policies. (j-l) Forecasting error for each $s(i)$ and each relative time index when $Z=3$, under task-agnostic (top) and weighted (bottom) policy, respectively.}


%\subsection{Algorithm Instantiations and Benchmarks}
\tbf{Algorithms and Benchmarks.} We test the above metrics on the following algorithms, which represent various instantiations of Alg. \ref{alg:codesign} for different $\lambdaforecast$ as well as today's prevailing method of optimizing for prediction MSE. Our algorithms and benchmarks are:
1) \textbf{Fully Task-aware ($\lambdaforecast=0$):} We co-design with $\lambdaforecast=0$ according to Alg. \ref{alg:codesign} to assess the full gains of compression.
2) \textbf{Weighted}: We instantiate Alg. \ref{alg:codesign} with $\lambdaforecast > 0$ to assess the benefits of task-aware compression as well as forecasting errors induced by compression. In practice, $\lambdaforecast$ is user-specified. For visual clarity, we show results for $\lambdaforecast=1$ in Fig. \ref{fig_real} since the trends for other $\lambdaforecast$ mirror those in Fig. \ref{fig:main_pca_full}.
3) \textbf{Task-agnostic (MSE)}: Our benchmark learns a forecast $\boldshat$ to minimize MSE prediction error, which is directly passed to the controller \textit{without} any co-design.

\textbf{Forecaster and Controller Models.} 
We compared forecast encoder/decoders with long short term memory (LSTM) DNNs \cite{LSTM} and simple feedforward networks. We observed similar performance for all models, which we hypothesize is because co-design needs to represent only a small set of \textit{control-relevant} features. We used standard DNN architectures, hyperparameters, and the Adam optimizer, as further detailed in the supplement. 
Our code and data are publicly available at \url{https://github.com/chengjiangnan/cooperative_networked_control}.
% \SC{All our code and data (provided in the supplement) will be \textbf{publicly-released} after peer review.}
%Finally, we use differentiable quadratic program solvers to obtain the gradient of MPC's control cost using the \textsc{qpth} \cite{amos2017optnet} library. 

\subsection{Linear Dynamics}
\label{subsec:linear}

\begin{figure*}[t]
\vskip 0.2in
\begin{center}
\includegraphics[width=0.8\columnwidth]{figures/main_legend.pdf}
%\subfigure[]{
\subfigure{
{\includegraphics[width=0.31\columnwidth]{figures/iot/iot_cost_bottleneck.pdf}}
\label{fig_iot_cost_bottleneck}
}
%\subfigure[]{
\subfigure{
{\includegraphics[width=0.31\columnwidth]{figures/cell/cell_cost_bottleneck.pdf}}
\label{fig_cell_cost_bottleneck}
}
%\subfigure[]{
\subfigure{
{\includegraphics[width=0.31\columnwidth]{figures/pjm/pjm_cost_bottleneck.pdf}}
\label{fig_pjm_cost_bottleneck}
}
%\subfigure[]{
\subfigure{
{\includegraphics[width=0.31\columnwidth]{figures/iot/control_errors.pdf}}
\label{fig_iot_control_errors}
}
%\subfigure[]{
\subfigure{
{\includegraphics[width=0.31\columnwidth]{figures/cell/control_errors.pdf}}
\label{fig_cell_control_errors}
}
%\subfigure[]{
\subfigure{
{\includegraphics[width=0.31\columnwidth]{figures/pjm/control_errors.pdf}}
\label{fig_pjm_control_errors}
}
%\subfigure[]{
\subfigure{
{\includegraphics[width=0.31\columnwidth]{figures/iot/forecast_errors_time_horizon.pdf}}
\label{fig_iot_forecast_errors_time_horizon}
}
%\subfigure[]{
\subfigure{
{\includegraphics[width=0.31\columnwidth]{figures/cell/forecast_errors_time_horizon.pdf}}
\label{fig_cell_forecast_errors_time_horizon}
}
%\subfigure[]{
\subfigure{
{\includegraphics[width=0.31\columnwidth]{figures/pjm/forecast_errors_time_horizon.pdf}}
\label{fig_pjm_forecast_errors_time_horizon}
}
\caption{\textbf{Real-world dataset results: } From left to right, the columns correspond to smart factory regulation from IoT sensors, taxi dispatching with cell demand, and battery storage optimization. \textbf{(Row 1)} Co-design achieves lower cost $\Jcontrol$ for smaller bottlenecks $\zbottleneck$ compared to task-agnostic methods. \textbf{(Row 2)} We also achieve lower error for each dimension $i$ of the vector control, $u(i)$, plotted for a highly-compressed $Z=3$. \textbf{(Row 3)} Co-design heavily reduces forecasting errors for initial horizons that are especially important for MPC's decision-making.}
%While a task-agnostic approach (blue) distributes forecasting errors roughly evenly across a horizon $H$, our co-design approaches heavily reduce errors for initial horizons that are especially important for MPC's decision-making.}
\label{fig_real}
\end{center}
\vskip -0.2in
\end{figure*}
%Aggregated forecasting error for  each relative time index when $Z=3$, under different policies. (j-l) Forecasting error for each $s(i)$ and each relative time index when $Z=3$, under task-agnostic (top) and weighted (bottom) policy, respectively.
%\caption{Results for real data. Columns from left to right corresponds to smart home regulation, taxi dispatching and battery storage optimization, respectively. (a-c) Control cost $J$ under different bottleneck dimension $Z$ and training policies; (d-f) Control error for each $u(i)$ when $Z=3$ under different training policies; (g-i) Aggregated forecasting error for  each relative time index when $Z=3$, under different policies. (j-l) Forecasting error for each $s(i)$ and each relative time index when $Z=3$, under task-agnostic (top) and weighted (bottom) policy, respectively.}


We now evaluate our algorithms on the IoT, taxi scheduling, and battery charging scenarios described in Sec. \ref{sec:scenarios}. Our results on a \textit{test} dataset are depicted in Fig. \ref{fig_real}, where each column corresponds to a real dataset and each row corresponds to an evaluation metric, as discussed below.

\textbf{How does compression affect control cost?} The first row of Fig. \ref{fig_real} quantifies the control cost $\Jcontrol$ for various compressed representations $\zbottleneck$. The optimal cost, in a dashed black line, is an unrealizable lower-bound cost when the controller is given the true future $s_{t:t+H-1}$ without any forecast error. The vertical bars show the distribution of costs across several test rollouts, each with different timeseries $\bolds$. 
%\JC{The vertical bars indicate the distribution of costs across different testing samples.}
%Our key result is that our co-design schemes (green and orange) achieve within \SC{$5 \%$} of the optimal cost but with a representation size that is at least \SC{$4\times$} smaller than a task-agnostic baseline in blue. Thus, for only \SC{$5 \%$} more cost than the optimal scheme, we achieve a compression gain of \SC{$15 \times, 15 \times$}, and \SC{$96 \times$} for the IoT, traffic, and battery datasets respectively. 
% In all curves, our weighted co-design approach (green) requires a marginally larger bottleneck than the purely task-aware approach ($\lambdaforecast = 0$) since the latent representation should minimize both control and forecast error. 
%The benefits of a weighted approach to also minimize forecast error are described subsequently.
Our key result is that our task-aware scheme (orange) achieves within $5 \%$ of the optimal cost, but with a small bottleneck size $Z$ of $4$, $4$ and $2$ for the IoT, traffic, and battery datasets, respectively. This corresponds to an absolute compression gain of $15\times$, $15\times$, and $96\times$ for each dataset. In contrast, with the same bottleneck sizes, a competing task-agnostic scheme (blue) incurs at least $25\%$ more control cost than our method.

Moreover, for the IoT and battery datasets, the task-agnostic benchmark requires a large bottleneck of $Z=35$ and $Z=11$, leading our approach to transmit $88 \%$ and $82 \%$ less data respectively. Strikingly, even for a large representation of $Z = 60$, a task-agnostic scheme incurs $100\%$ more cost than the optimal for the cell traffic dataset. This is because the cost function is highly sensitive to shortages with $\gamma_s \gg \gamma_e$, which is not captured by simply optimizing for \textit{mean} error. To clearly see the trend in Fig. \ref{fig_real}, we only plot until $Z=9$, but ran the experiments until $Z=60$. Our weighted approach (green) requires a marginally larger representation than the purely task-aware approach ($\lambdaforecast = 0$) since it should minimize both control and forecast error. 

\textbf{Does co-design reduce control errors?}
We now investigate how the compression benefits of co-design arise. Given the stochastic nature of all our real world datasets, all prediction models inevitably produce forecasting error, which in turn induce errors in selecting controls. However, the key benefit of co-design methods is they explicitly model and account for how MPC chooses controls based on noisy forecasts $\boldshat$, and are thus able to minimize the control error, which we now quantify.

As defined in Sec. \ref{subsec:input_driven_LQR}, for any state $x_t$, $u_t$ is the optimal first MPC control given perfect knowledge of $s_{t:t+H-1}$, while $\hat{u}_t$ is MPC's actual enacted control given a noisy forecast.
%First, we define the optimal control $u_t$ to be the control selected by MPC given the ground-truth future $s_{t:t+H-1}$ from a given initial state $x_t$. For the same initial condition $x_t$, we define $\hat{u}_t$ to be MPC's control when it is given a forecast $\shat_{t:t+H-1}$ which can have errors due to encoding/decoding. 
Then, the \textit{control errors} across various control dimensions $i$ are the MSE error $\doublevert u_t(i)  - \hat{u}_t(i) \doublevert^2_2$ between optimal control $u_t(i)$ and $\hat{u}_t(i)$. The second row of Fig. \ref{fig_real} clearly shows that our task-aware and weighted methods (orange and green) achieve lower \textit{control} error on all three datasets.

\textbf{Why does co-design yield task-relevant forecasts?} 
To further show that our co-design approach reduces forecasting error for the purposes of an ultimate control task, we show forecasting errors across various time horizons in the third row of Fig. \ref{fig_real}. As argued in the previous section, all forecasting models produce prediction error. However, a task-agnostic forecast (blue) roughly equally distributes prediction error across the time horizon $t$ to $t+H-1$. In stark contrast, the weighted co-design approach (green) drastically reduces prediction errors in the \textit{near future} since MPC enacts the first control $u_t$ and then re-plans on a rolling horizon. Of course, the full forecast $\shat_{t:t+H-1}$ matters to enact control plan $\hat{u}_{t:t+H-1}$, but the cost is most sensitive to the initial forecast and control errors in our MPC scenarios. For visual clarity, we present forecast errors of the fully task-aware approach ($\lambdaforecast = 0$) in the supplement, since the errors are much larger than the other two methods. 
%\SC{discuss 12 hours period battery.}

\subsection{Nonlinear Dynamics with Transition Noise}
\label{subsec:nonlinear}

% \begin{figure}[ht]
% \vskip 0.2in
% \begin{center}
% \subfigure{
% {\includegraphics[width=0.6\columnwidth]{figures/video/video_cost_bottleneck.pdf}
% }}
% \caption{Co-Design Results with \textit{Nonlinear} Dynamics and Transition Noise.}
% \label{fig:nonlinear}
% \end{center}
% \vskip -0.2in
% \end{figure}

\begin{wrapfigure}{R}{0.5\columnwidth}
\centering
\includegraphics[width=0.5\columnwidth]{figures/video/video_cost_bottleneck.pdf}
\caption{Co-Design Results with \textit{Nonlinear} Dynamics and Transition Noise.}
\label{fig:nonlinear}
% \vskip -0.5em
\end{wrapfigure}


To illustrate that our co-design approach works well for systems with \textbf{nonlinear and stochastic} dynamics, we also provide a nonlinear
example concerning an idealized mobile video streaming scenario. In this application, a mobile video client stores a buffer of video segments and must choose a video quality to download for the next segment of video. The goal is to maximize the quality of video while minimizing video stalls, which occur when the buffer under-flows while waiting for a segment to be downloaded. Here, state $x_t$ represents the buffer of stored video segments, control $u_t$ is segment quality, and $s_t$ is network throughput. The nonlinear dynamics are $x_{t+1} = [ x_{t} - u_t \oslash s_t ]_+ + L_x + \eta_{t}$, where $\oslash$ represents element-wise division, $L_x$ is the increase in stored video for each download, and $\eta_t$ is Gaussian transition noise. The cost aims to keep a positive buffer and have high video quality:
    $\Jcontrol(\boldx, \boldu) = \sum_{t=0}^{T} \gamma_x \doublevert x_t - L_x \doublevert^2_2  + \sum_{t=0}^{T-1} \gamma_u \doublevert u_t - L_u\doublevert^2_2.$

Fig. \ref{fig:nonlinear} clearly shows our approach works quite well for a nonlinear scenario with transition noise, which complements the three linear examples in Sec. \ref{subsec:linear}. In the above experiments, the parameters are: $T=60$, $W = H = 15$, $m=n=p=4$, $\gamma_x = 0.25$, $\gamma_u = 1$, $L_x = 0.5 \times \mathbbm{1}_n$, $L_u = 0.2 \times \mathbbm{1}_m$. 

\textbf{Limitations: } \SC{Our work does not automatically learn the optimal bottleneck size $Z$ that minimizes control cost nor necessarily learn a human-interpretable latent representation.} 
%Further, co-design may yield minimal benefits if every timeseries feature is unique and equally-important for control.}
%However, as evidenced by our evaluation, such a scenario is unlikely for diverse engineering domains.}



\section{Conclusion}
\label{sec:conclusion}
Society is rapidly moving towards ``smart cities'' \cite{batty2012smart,al2015applications}, where smart grid and 5G wireless network operators alike can share forecasts to enhance external control applications. This paper presents a preliminary first step towards this goal, by contributing an algorithm to learn task-relevant, compressed representations of timeseries for a control objective. 
%On real datasets spanning diverse engineering domains, our method transmits at least $80\%$ less data and achieves better control performance than today's task-agnostic benchmarks.
Our future work will center around privacy guarantees that constrain learned representations to filter personal features, such as individual mobility patterns. Further, we want to certify our algorithm does not reveal proprietary control logic or private internal states of the downstream controller. While recent work has addressed how to value datasets for supervised learning \cite{ghorbani2019data,agarwal2019marketplace}, a promising extension of our work is to price timeseries datasets for cooperative control in a data-market. Indeed, our ability to gracefully trade-off control cost with data exchange lends itself to an economic analysis. %Overall, we believe our work is a valuable first step to learning efficient representations for networked control.
%, which is timely given recent deployments of fleets of IoT devices, 5G networks, and diverse energy sources in increasingly ``smart'' cities.


\section{Funding Disclosure}
\label{sec:funding}
This material is based upon work supported by the National Science Foundation under Grant No. 2133481 and 2133403. Any opinions, findings, and conclusions or recommendations expressed in this material are those of the author(s) and do not necessarily reflect the views of the National Science Foundation.

\bibliography{ref/ref}
\bibliographystyle{unsrtnat}

\newpage
\section*{Checklist}
%%% BEGIN INSTRUCTIONS %%%
% The checklist follows the references.  Please
% read the checklist guidelines carefully for information on how to answer these
% questions.  For each question, change the default \answerTODO{} to \answerYes{},
% \answerNo{}, or \answerNA{}.  You are strongly encouraged to include a {\bf
% justification to your answer}, either by referencing the appropriate section of
% your paper or providing a brief inline description.  For example:
% \begin{itemize}
%   \item Did you include the license to the code and datasets? \answerYes{See Section~\ref{gen_inst}.}
%   \item Did you include the license to the code and datasets? \answerNo{The code and the data are proprietary.}
%   \item Did you include the license to the code and datasets? \answerNA{}
% \end{itemize}
% Please do not modify the questions and only use the provided macros for your
% answers.  Note that the Checklist section does not count towards the page
% limit.  In your paper, please delete this instructions block and only keep the
% Checklist section heading above along with the questions/answers below.
%%% END INSTRUCTIONS %%%

\begin{enumerate}

\item For all authors...
\begin{enumerate}
  \item Do the main claims made in the abstract and introduction accurately reflect the paper's contributions and scope?
      \answerYes{\textbf{Please see Section \ref{sec:intro}.}}
  \item Did you describe the limitations of your work?
      \answerYes{\textbf{Please see the end of the evaluation (Section \ref{sec:evaluation}).}}
  \item Did you discuss any potential negative societal impacts of your work?
      \answerYes{\textbf{The work aims to provide positive societal impacts by optimizing social systems. We discussed privacy as a future step in Section \ref{sec:conclusion}.}}
  \item Have you read the ethics review guidelines and ensured that your paper conforms to them?
    \answerYes{}
\end{enumerate}

\item If you are including theoretical results...
\begin{enumerate}
  \item Did you state the full set of assumptions of all theoretical results?
      \answerYes{\textbf{Please see Section \ref{sec:algorithm}.}}
    \item Did you include complete proofs of all theoretical results?
        \answerYes{\textbf{Please see Section \ref{sec:algorithm}.}}
\end{enumerate}

\item If you ran experiments...
\begin{enumerate}
  \item Did you include the code, data, and instructions needed to reproduce the main experimental results (either in the supplemental material or as a URL)?
    \answerYes{}
  \item Did you specify all the training details (e.g., data splits, hyperparameters, how they were chosen)?
      \answerYes{\textbf{Please see Appendix Section \ref{sec:appendix_evaluation}.}}
    \item Did you report error bars (e.g., with respect to the random seed after running experiments multiple times)?
        \answerYes{\textbf{Please see the plots in Section 5.}}
    \item Did you include the total amount of compute and the type of resources used (e.g., type of GPUs, internal cluster, or cloud provider)?
        \answerYes{\textbf{Please see Appendix Section \ref{sec:appendix_evaluation}.}}
\end{enumerate}

\item If you are using existing assets (e.g., code, data, models) or curating/releasing new assets...
\begin{enumerate}
  \item If your work uses existing assets, did you cite the creators?
      \answerYes{\textbf{Please see the citations for all three publicly-available datasets in Section \ref{sec:scenarios}.}} 
  \item Did you mention the license of the assets?
      \answerYes{\textbf{As stated in the Appendix, the licenses for the individual assets were not given online.}}
  \item Did you include any new assets either in the supplemental material or as a URL?
      \answerYes{\textbf{We provide our collected IoT dataset in the supplement and will make it publicly-available online.}}
  \item Did you discuss whether and how consent was obtained from people whose data you're using/curating?
      \answerYes{\textbf{Please see Appendix Section \ref{sec:appendix_evaluation}.}}
  \item Did you discuss whether the data you are using/curating contains personally identifiable information or offensive content?
      \answerYes{\textbf{Yes, we stated in Appendix Section \ref{sec:appendix_evaluation} that there no is no personally identifiable or private information.}}
\end{enumerate}

\item If you used crowdsourcing or conducted research with human subjects...
\begin{enumerate}
  \item Did you include the full text of instructions given to participants and screenshots, if applicable?
    \answerNA{}
  \item Did you describe any potential participant risks, with links to Institutional Review Board (IRB) approvals, if applicable?
    \answerNA{}
  \item Did you include the estimated hourly wage paid to participants and the total amount spent on participant compensation?
    \answerNA{}
\end{enumerate}

\end{enumerate}


\newpage
\appendix
\section{Appendix}

% \subsection{Nonlinear Dynamics with Transition Noise }
% \label{subsec:nonlinear}

% To illustrate that our co-design approach works well for systems with \textbf{nonlinear} dynamics, we provide the following nonlinear
% example concerning an idealized mobile video streaming scenario. In this application, a mobile video client stores a buffer of video segments and must choose a video quality to download for the next segment of video. The goal is to maximize the quality of video while minimizing video stalls, which occur when the buffer under-flows while waiting for a segment to be downloaded. Here, state $x_t$ represents the buffer of stored video segments, control $u_t$ is segment quality, and $s_t$ is network throughput. The nonlinear dynamics are $x_{t+1} = [ x_{t} - u_t \oslash s_t ]_+ + L_x + \eta_{t}$, where $\oslash$ represents element-wise division, $L_x$ is the increase in stored video for each download, and $\eta_t$ is Gaussian transition noise. The cost aims to keep a positive buffer and have high video quality:
%     $\Jcontrol(\boldx, \boldu) = \sum_{t=0}^{T} \gamma_x \doublevert x_t - L_x \doublevert^2_2  + \sum_{t=0}^{T-1} \gamma_u \doublevert u_t - L_u\doublevert^2_2.$

% % \begin{figure}[ht]
% \vskip 0.2in
% \begin{center}
% \subfigure{
% {\includegraphics[width=0.6\columnwidth]{figures/video/video_cost_bottleneck.pdf}
% }}
% \caption{Co-Design Results with \textit{Nonlinear} Dynamics and Transition Noise.}
% \label{fig:nonlinear}
% \end{center}
% \vskip -0.2in
% \end{figure}

\begin{wrapfigure}{R}{0.5\columnwidth}
\centering
\includegraphics[width=0.5\columnwidth]{figures/video/video_cost_bottleneck.pdf}
\caption{Co-Design Results with \textit{Nonlinear} Dynamics and Transition Noise.}
\label{fig:nonlinear}
% \vskip -0.5em
\end{wrapfigure}

% Fig. \ref{fig:nonlinear} clearly shows our approach works quite well for a nonlinear scenario with transition noise, which complements the three diverse examples in the main paper. In the above experiments, the parameters are: $T=60$, $W = H = 15$, $m=n=p=4$, $\gamma_x = 0.25$, $\gamma_u = 1$, $L_x = 0.5 \times \mathbbm{1}_n$, $L_u = 0.2 \times \mathbbm{1}_m$. 

\subsection{Time Horizon} 
\begin{figure}[ht]
\vskip 0.2in
\begin{center}
\subfigure{
{\includegraphics[width=0.6\columnwidth]{figures/time_horizon/time_horizon.pdf}}
}
\caption{Time horizon illustration.}
\label{fig:time_horizon}
\end{center}
\vskip -0.2in
\end{figure}

Fig. \ref{fig:time_horizon} illustrates the time horizon of the problem we consider in Sec. \ref{sec:problem_statement}.

\subsection{Input-driven LQR with Transition Noise} \label{subsec:lqr_transition_noise}
Suppose at each time interval $t$, we add a noise vector $w_t \in \reals^{n}$ with mean zero and covariance $\Sigma_{ww}$ to the dynamics:
\begin{align}
x_{t+1} = A x_t + B u_t + C s_t + w_t,
\end{align}
Then Eq. \ref{eq:rewrite_x_t} becomes
\begin{align}
    & x_{i+1} = A^{i+1} x_0 + \bm{M_{i}} \boldu + \bm{N_{i}} \bolds + \bm{P_i} \boldw
\end{align}
where $\boldw := w_{0:H-1}$ and $\bm{P_{i}} = \begin{bmatrix}
A^{i} & A^{i-1} & \cdots & I & \bm{0}
\end{bmatrix} \in \mathbb{R}^{n \times nH}$.

And hence Eq. \ref{eq:quadratic_jmpc} becomes
\begin{align}
\JMPC(\boldu; x_0, \bolds) ~=~ &  \boldu^\top (\underbrace{\bm{R} + \sum_{i=0}^{H-1} \bm{M_{i}}^\top Q \bm{M_{i}}}_{\bm{K}}) \boldu + 2[\underbrace{\sum_{i=0}^{H-1} \bm{M_{i}}^\top Q (A^{i+1} x_0 + \bm{N_{i}} \bolds + \bm{P_i} \boldw)}_{\bm{k}(x_0, \bolds)}]^\top \boldu + \\
& \underbrace{\sum_{i=0}^{H-1} (A^{i+1} x_0 + \bm{N_{i}} \bolds + \bm{P_i} \boldw)^\top Q (A^{i+1} x_0 + \bm{N_{i}} \bolds + \bm{P_i} \boldw)}_{\text{independent of $\boldu$}},
\end{align}
that is,
\begin{align}
\expec_\boldw[\JMPC(\boldu; x_0, \bolds)] ~=~ &  \boldu^\top (\underbrace{\bm{R} + \sum_{i=0}^{H-1} \bm{M_{i}}^\top Q \bm{M_{i}}}_{\bm{K}}) \boldu + 2[\underbrace{\sum_{i=0}^{H-1} \bm{M_{i}}^\top Q (A^{i+1} x_0 + \bm{N_{i}} \bolds)}_{\bm{k}(x_0, \bolds)}]^\top \boldu + \\
& \underbrace{\sum_{i=0}^{H-1} (A^{i+1} x_0 + \bm{N_{i}} \bolds)^\top Q (A^{i+1} x_0 +\bm{N_{i}} \bolds) + \expec_\boldw[(\bm{P_i} \boldw)^\top Q \bm{P_i} \boldw] }_{\text{independent of $\boldu$}}.
\end{align}

Notice that $\boldw$ only affects the constant term, which is independent of $\boldu$. Therefore, the analysis after Eq. \ref{eq:quadratic_jmpc} still holds.

% So $\boldu^*$ is independent of the noise $\boldw$ and the variance of $x_0$. $\JMPC$ is independent of $\E[x_0]$ but positively correlated to the variance of $x_0$.

% \subsection{Expectation of $x_0$}

% If the system is fully observable, then we know the exact the value of $x_0$, which will be $\E[x_0]$.

% If the system is not fully observable, then 
% $y_i = D x_i + v_i$

% Nee to find the estimate of $x_i$ with the smallest variation under the given observations. So
% $\E[x_{i+1}] = A \E[x_{i+1}] + B u_i + C s_i + L_i(y_i - D(A \E[x_{i+1}] + B u_i + C s_i))$
% where $L_i = AP_{i}D^\top(DP_iD^\top + \Sigma_{vv})^{-1}$ and 
% $P_{i+1} = A(P_i - P_i D^\top (D P_i D^\top + \Sigma_{vv})^{-1} DP_i)A^\top + \Sigma_{ww}$, and $P_0$ is the covariance matrix of $x_0$.

\subsection{Additional Explanations on the Proof of Proposition \ref{prop:input_dirven_lqr}}
Here, we provide some additional explanations on the proof of Proposition \ref{prop:input_dirven_lqr}, which are not included in the main paper due to space limits. 

1) \textbf{Positive definite matrix.} $\Psi + \lambdaforecast I$ ($\lambdaforecast > 0$) is positive definite because, $(\Psi + \lambdaforecast I)^\top = \Psi + \lambdaforecast I$, and $\Joverall \geq \lambdaforecast || \boldshat - \bolds||_2^2 > 0$ for any $\boldshat \neq \bolds$.

2) \textbf{Eigen-decomposition.}
The eigen-decomposition of $\Psi + \lambdaforecast I$ is $Y \Lambda Y^{-1}$, where $Y \in \reals^{\pH \times \pH}$ and the columns of $Y$ are the normalized eigen-vectors of $\Psi + \lambdaforecast I$, and $\Lambda \in \reals^{\pH \times \pH}$ is the diagonal matrix whose diagonal elements are the eigenvalues of $\Psi + \lambdaforecast I$. Since $\Psi + \lambdaforecast I$ is symmetric, $Y$ is also orthogonal, i.e., $Y^{-1} = Y^\top$. So $Y \Lambda Y^{-1} = Y \Lambda Y^\top$.

3) \textbf{Inverse matrix.} The matrix $\Lambda^{\frac{1}{2}}Y^\top$ is invertible because $\Psi + \lambdaforecast I$ is positive definite and its eigenvalues are all positive.

% 4) \textbf{Optimality.} If there is another encoder and decoder pair, denoted by ($E'$,$D'$), that achieves a smaller $\Joverall$ than ($E = \Omega Y^\top \Lambda^{\frac{1}{2}}$, $D = (Y^\top \Lambda^{\frac{1}{2}})^{-1}\Omega^\top $), then the projection $Y^\top \Lambda^{\frac{1}{2}})^{-1}E'$ and the inverse projection $Y^\top \Lambda^{\frac{1}{2}}D'$ give a solution to the PCA problem with smaller reconstruction error. This is contradictory to the assumption that $\Omega$ and $\Omega^\top$ are the optimal projection and inverse projection.

\subsection{Details on the LQR Simulations}

Here we provide further details on the two LQR simulations mentioned in Sec. \ref{subsec:input_driven_LQR}. In both of the simulations, vector timeseries $s$ has log, negative exponential, sine, square, and saw-tooth functions superimposed with a Gaussian random walk noise process.


\subsubsection{Basic LQR Simulation (Fig. \ref{fig:main_pca_full})}

\begin{figure*}[t]
\vskip 0.2in
\begin{center}

\includegraphics[width=0.8\columnwidth]{figures/pca_legend.pdf}

\subfigure{
{\includegraphics[width=0.4\columnwidth]{figures/pca_full/pca_full_cost_bottleneck.pdf}}
\label{fig_pca_full_cost_bottleneck}
}
\subfigure{
{\includegraphics[width=0.4\columnwidth]{figures/pca_full/pca_full_ctrl_MSE.pdf}}
\label{fig_pca_full_ctrl_MSE}
}
\subfigure{
{\includegraphics[width=0.4\columnwidth]{figures/pca_full/pca_full_fcst_MSE.pdf}}
\label{fig_pca_full_fcst_MSE}
}
\subfigure{
{\includegraphics[width=0.4\columnwidth]{figures/pca_full/pca_cost_bottleneck_combined.pdf}
\label{fig_pca_full_forecast_errors}
}
}
\caption{\textbf{Analytic results for linear control:} (a) By only representing information salient to a control task, our co-design method (orange) achieves the optimal control cost with  
$43\%$
% $1.75\times$ 
less data than a standard MSE approach (``task-agnostic'', blue). Formal definitions of all benchmarks are in Sec. \ref{sec:evaluation}. (b-c) By weighting prediction error by $\lambdaforecast > 0$, we learn representations that are compressible, have good predictive power, and lead to near-optimal control cost (\eg~ $\lambdaforecast=1.0$).
(d) For the \textit{same} timeseries $\bolds$, two different control tasks require various amounts of data shared, motivating our task-centric representations.}
% \caption{\textbf{Analytic Results for Linear Control:} (a) By compressing information salient to a control task, our co-design method (orange) can effectively lower the control cost than a standard RMSE approach (``task-agnostic'', blue). Formal definitions of all benchmarks are in Sec. \ref{sec:evaluation}. (b-c) By weighting prediction error by $\lambdaforecast > 0$, we learn representations that are compressible, have good predictive power, and lead to near-optimal control cost (\eg $\lambdaforecast=1.0$).
% (d) For the \textit{same} timeseries $\bolds$ but a different control task (bottom), co-design method may not differ too much from the standard RMSE approach.}
\label{fig:main_pca_full}
\end{center}
\vskip -0.2in
\end{figure*}


1) \textbf{Dynamics:}
\begin{align*}
x_{t+1} = x_t + u_t - C s_t
\end{align*}
2) \textbf{Cost function:}
\begin{align*}
\JMPC = \frac{1}{1000} (\sum_{t=0}^{H} ||x_t||^2_2 + \sum_{t=0}^{H-1} ||u_t||^2_2) 
\end{align*}
3) \textbf{Parameters:} $H = 20$; $n = m = p = 5$; $C=\text{diag}(1, 2, \cdots, 5)$ for Fig. \ref{fig_pca_full_cost_bottleneck}-\ref{fig_pca_full_fcst_MSE} and Fig. \ref{fig_pca_full_forecast_errors} top, and $C=\text{diag}(1.5, 2, \cdots, 3.5)$ for Fig. \ref{fig_pca_full_forecast_errors} bottom.

As per Proposition \ref{prop:input_dirven_lqr}, we solve a simple low-rank approximation problem per bottleneck $Z$ to obtain the optimal encoder $E$, decoder $D$, and use Eqs. \ref{eq:low_rank_approximation}-\ref{eq:low_rank_approximation_final} to obtain the control and prediction costs. Clearly, our co-design algorithm (orange) outperforms a task-agnostic approach (blue) that simply optimizes for MSE.

\subsubsection{LQR Simulation with MPC (Fig. \ref{fig:main_pca})}
\begin{figure*}[t]
\vskip 0.2in
\begin{center}

\includegraphics[width=0.8\columnwidth]{figures/pca_legend.pdf}

\subfigure{
{\includegraphics[width=0.4\columnwidth]{figures/pca/pca_mpc_cost_bottleneck.pdf}}
\label{fig_pca_cost_bottleneck}
}
\subfigure{
{\includegraphics[width=0.4\columnwidth]{figures/pca/pca_mpc_ctrl_MSE.pdf}}
\label{fig_pca_ctrl_MSE}
}
\subfigure{
{\includegraphics[width=0.4\columnwidth]{figures/pca/pca_mpc_fcst_MSE.pdf}}
\label{fig_pca_fcst_MSE}
}
\subfigure{
{\includegraphics[width=0.4\columnwidth]{figures/pca/pca_cost_bottleneck_combined.pdf}
\label{fig_pca_forecast_errors}
}
}
\caption{\textbf{Linear control with MPC:} We repeat our analysis of input-driven LQR, but solve the problem in a receding horizon manner with forecasts for $H < T$ as discussed in Section \ref{subsec:input_driven_LQR} and Figure \ref{fig:main_pca_full}.
    (a) By only representing information salient to a control task, our co-design method (orange) achieves the optimal control cost with 
$60\%$ less data than a standard MSE approach (``task-agnostic'', blue). Formal definitions of all benchmarks are in Sec. \ref{sec:evaluation}. (b-c) By weighting prediction error by $\lambdaforecast > 0$, we learn representations that are compressible, have good predictive power, and lead to near-optimal control cost (\eg~ $\lambdaforecast=1.0$). The forecasting error of the task-aware scheme (orange) is much larger than the rest and thus not shown in the zoomed-in view.
(d) For the \textit{same} timeseries $\bolds$, two different control tasks require various amounts of data shared, motivating our task-centric representations.}
\label{fig:main_pca}
\end{center}
\vskip -0.2in
\end{figure*}


1) \textbf{Dynamics:}
\begin{align*}
x_{t+1} = x_t + u_t - C s_t
\end{align*}
2) \textbf{Cost function:}
\begin{align*}
\JMPC = \sum_{t=0}^{T} ||x_t||^2_2 + \sum_{t=0}^{T-1} ||u_t||^2_2 
\end{align*}
3) \textbf{Parameters:} $T = 100$, $W = H = 15$; $n = m = p = 5$; $C=\text{diag}(1, 2, \cdots, 5)$ for supplement Fig. \ref{fig_pca_cost_bottleneck}-\ref{fig_pca_fcst_MSE} and Fig. \ref{fig_pca_forecast_errors} top, and $C=\text{diag}(3, 3, \cdots, 3)$ for Fig. \ref{fig_pca_forecast_errors} bottom.

% We also run another LQR simulation with MPC, as shown in Fig. \ref{fig:main_pca}. The dynamics and the cost function are as follows:
% \begin{align*}
% & x_{t+1} = x_t + u_t - C s_t, \\
% & \JMPC = \sum_{t=0}^{T} ||x_t||^2_2 + \sum_{t=0}^{T-1} ||u_t||^2_2 
% \end{align*}

% The parameters are: 
% \begin{itemize}
% \item $T = 100$, $W = H = 15$, $n = m = p = 5$. 
% \item $C=\text{diag}(1, 2, \cdots, 5)$ for Fig. \ref{fig_pca_cost_bottleneck}-\ref{fig_pca_fcst_MSE} and Fig. \ref{fig_pca_forecast_errors} top; $C=\text{diag}(3, 3, \cdots, 3)$ for Fig. \ref{fig_pca_full_forecast_errors} bottom.
% \end{itemize}

% The data is the same as described in the last section.

\subsection{IoT Data Collection}
\begin{figure}[ht]
\vskip 0.2in
\begin{center}
\subfigure{
\raisebox{10mm}{\includegraphics[width=0.2\columnwidth]{figures/appendix/iot/sensor/IoT_sensor.png}}
}
\subfigure{
{\includegraphics[width=0.6\columnwidth]{figures/appendix/iot/sensor/example_plot.pdf}}
}
\caption{Environmental sensor on the Google Edge TPU (left) and example stochastic timeseries (right).}
\label{fig:iot_sensor}
\end{center}
\vskip -0.2in
\end{figure}

Fig. \ref{fig:iot_sensor} shows the environmental sensor board (connected to an Edge TPU DNN accelerator) and an example of collected stochastic timeseries for our IoT data.

\subsection{Detailed Evaluation Settings}
\label{sec:appendix_evaluation}
\begin{figure}[ht]
\vskip 0.2in
\begin{center}
\centerline{\includegraphics[width=0.8\columnwidth]{figures/appendix/model/lstm.pdf}}
\caption{LSTM timeseries network} 
\label{fig:lstm}
\end{center}
\vskip -0.2in
\end{figure}

\begin{figure}[ht]
\vskip 0.2in
\begin{center}
\centerline{\includegraphics[width=0.5\columnwidth]{figures/appendix/model/DNN.pdf}}
\caption{2-hidden-layer feedforward network} 
\label{fig:DNN}
\end{center}
\vskip -0.2in
\end{figure}

We now provide further details on Sec. \ref{sec:evaluation} by summarizing the settings of our evaluation.

\subsubsection{Forecaster, Controller $\&$ Data Scaling}

\textbf{Basic Forecaster Settings.} In all three scenarios, the encoder parameters $\thetaencoder$ are responsible for \textit{both} forecasting and compression. We first have a forecasting model that first provides a full-dimensional forecast $\tilde{s}_{t:t+H-1}$, and then adopts simple linear encoder $E \in \reals^{\zbottleneck \times \pH}$ to yield $\phi_t$. The combination of the forecasting model's parameters and encoder $E$ constitute $\thetaencoder$. Then, a linear decoder $D \in \reals^{\pH \times \zbottleneck}$eventually produces decoded forecast $\hat{s}_{t:t+H-1}$. The model used to provide full-dimensional forecast $\tilde{s}_{t:t+H-1}$ varies case by case, as described subsequently.

\textbf{Smart Factory Regulation with IoT Sensors.} For forecasting, we adopt an LSTM timeseries network, as shown in Fig. \ref{fig:lstm}, with $W+H-2$ cells and hidden size $64$. The parameters associated with the forecaster and controller are set as follows: $T=72$, $W = H = 15$, $n = m = p = 4$; $u_\text{min} = -0.95 \times \mathbbm{1}_{p}, u_\text{max} = 0.95 \times \mathbbm{1}_{p}$, $\gamma_e = \gamma_s = \gamma_u = 1$. Further, we scale $s_t(i)$ to be within $[-1, 1], \forall i$.

\textbf{Taxi Dispatch Based on Cell Demand Data.} For forecasting, we adopt a 2-hidden-layer feedforward network, as shown in Fig. \ref{fig:DNN}, with hidden size $64$ and ReLu activation. The parameters associated with the forecaster and controller are set as follows: $T=32$, $W = H = 15$, $n = m = p = 4$; no constraint on $u_t$, and $\gamma_e = 1, \gamma_s = 100, \gamma_u = 1$. Further, we scale $s_t(i)$ to be within $[0, 1], \forall i$. 

\textbf{Battery Storage Optimization.} For forecasting, we adopt a 2-hidden-layer feedforward network, as shown in Fig. \ref{fig:DNN}, with hidden size $64$ and ReLu activation. The parameters associated with the forecaster and controller are set as follows: $T=122$, $W = H = 24$, $n = m = p = 8$; no constraint on $u_t$, and $\gamma_e = \gamma_s = \gamma_u = 1$. Further, we scale $s_t(i)$ to be within $[0, 1], \forall i$.

We observed similar performance for feedforward networks and LSTMs since the crux of our problem is to find a small set of task-relevant features for control.

\subsubsection{Training}
\begin{table}[ht]
\caption{Train/Test Timeseries, Training Epochs and Runtime.}
\label{tab:eval}
\vskip 0.15in
\begin{center}
\begin{small}
\begin{sc}
\begin{tabular}{lcccr}
\toprule
Dataset & Train/Test & Training & Runtime\\
& Timeseries & Epochs & \\
\midrule
IoT & 30/30 & 1000 & $<96$ hrs\\
Cell & 17/17 & 1000 & $<48$ hrs\\
Battery & 15/15 & 2000 & $<1$ hr \\
\bottomrule
\end{tabular}
\end{sc}
\end{small}
\end{center}
\vskip -0.1in
\end{table}

Our evaluation runs on a Linux machine with 4 NVIDIA GPUs installed (3 Geforce and 1 Titan).
Our code is based on Pytorch. We use the Adam optimizer and learning rate $10^{-3}$ for all the evaluations. The number of train/test timeseries\footnote{With MPC, each timeseries corresponds to $T$ samples, such as $T=72$ for the IoT scenario.}, training epochs, and resulting runtime are summarized in Table \ref{tab:eval}. 
The IoT dataset is provided in our code release and it does not have any personally identifiable or private information. 
% We provide our IoT dataset in our code release and will make it publicly-available (without any restrictions or license) online after the peer review process. The dataset does not have any personally identifiable or private information. 
The publicly-available electricity and cellular datasets did not have a stated license online. 


% \subsection{Key Result}
% \begin{table}[ht]
% \caption{Additional cost for each scheme on different datasets with the same compression gain -- the largest compression gain when task-aware scheme requires no more than $5\%$ additional cost.}
% \label{tab:key_result}
% \vskip 0.15in
% \begin{center}
% \begin{small}
% \begin{sc}
% \begin{tabular}{lcccr}
% \toprule
% Dataset & Task & Task & Weighted \\
% (Comp. gain) & Agnostic& Aware& \\
% \midrule
% IoT ($15\times$) & $32.3\%$ & $4.55\%$ & $7.30\%$ \\
% Cell ($15\times$) & $1435\%$ & $1.97\%$ & $41.6\%$ \\
% PJM ($96\times$) & $32.2\%$ & $4.38\%$ & $8.56\%$ \\
% \bottomrule
% \end{tabular}
% \end{sc}
% \end{small}
% \end{center}
% \vskip -0.1in
% \end{table}

% Table \ref{tab:key_result} presents the additional cost for each scheme on different datasets with the same compression gain. When the compression gains on the IoT, Cell and PJM dataset are $15\times$, $15\times$ and $96\times$ respectively, task-aware scheme requires only no more than $5\%$ additional cost, while task-agnostic scheme yields at least $30\%$ of the additional cost (notably, $1435\%$ additional cost on the cell dataset). Our weighted scheme yields a additional cost larger than the task-aware scheme but still relatively much smaller than the task-agnostic scheme. 

\subsection{Further Analysis on the Evaluation Results}
For better understanding of the differences between different schemes, we give further analysis on our evaluation results in Sec. \ref{sec:evaluation}.

\begin{figure*}[t]
\vskip 0.2in
\begin{center}
%\subfigure[]{
\subfigure{
{\includegraphics[width=0.31\columnwidth]{figures/iot/forecast_errors.pdf}}
\label{fig_iot_forecast_errors}
}
%\subfigure[]{
\subfigure{
{\includegraphics[width=0.31\columnwidth]{figures/cell/forecast_errors.pdf}}
\label{fig_cell_forecast_errors}
}
%\subfigure[]{
\subfigure{
{\includegraphics[width=0.31\columnwidth]{figures/pjm/forecast_errors.pdf}}
\label{fig_pjm_forecast_errors}
}
\caption{\textbf{Forecasting error comparison: task-agnostic vs. weighted scheme.} From left to right, the columns correspond to smart factory regulation from IoT sensors, taxi dispatching with cell demand, and battery storage optimization. The heatmaps show how co-design minimizes errors on timeseries elements $s(i)$ and forecast horizons that are salient for the control task when $Z=3$.}
\label{fig_heatmap}
\end{center}
\vskip -0.2in
\end{figure*}
%Aggregated forecasting error for  each relative time index when $Z=3$, under different policies. (j-l) Forecasting error for each $s(i)$ and each relative time index when $Z=3$, under task-agnostic (top) and weighted (bottom) policy, respectively.
%\caption{Results for real data. Columns from left to right corresponds to smart home regulation, taxi dispatching and battery storage optimization, respectively. (a-c) Control cost $J$ under different bottleneck dimension $Z$ and training policies; (d-f) Control error for each $u(i)$ when $Z=3$ under different training policies; (g-i) Aggregated forecasting error for  each relative time index when $Z=3$, under different policies. (j-l) Forecasting error for each $s(i)$ and each relative time index when $Z=3$, under task-agnostic (top) and weighted (bottom) policy, respectively.}

\textbf{Why does co-design yield task-relevant forecasts? (Continued)}

We further contrast the prediction errors made by task-agnostic and co-design approaches in the heatmaps of Fig. \ref{fig_heatmap}. In each heatmap, the x-axis represents the future time horizon, while the y-axis represents forecasting errors across various dimensions of timeseries $s$, denoted by $s(i)$.  Clearly, a weighted approach significantly reduces prediction error for near time-horizons, which is most pronounced for the battery dataset.


\begin{figure}[ht]
\vskip 0.2in
\begin{center}
\includegraphics[width=0.8\columnwidth]{figures/forecast_legend.pdf}

\subfigure{
{\includegraphics[width=0.4\columnwidth]{figures/appendix/pjm/z_9/forecasts_no_task_aware_sample_1_time_84_signal_0.pdf}}
}
\subfigure{
{\includegraphics[width=0.4\columnwidth]{figures/appendix/pjm/z_9/forecasts_task_aware_sample_1_time_84_signal_0.pdf}}
}
\caption{\textbf{Forecast comparison: task-agnostic/weighted schemes vs. task-aware scheme.}  Example forecasts for the battery charging scenario at $t=84$ when $Z=9$, for both our task-agnostic/weighted schemes (left) and the task-aware scheme (right). Clearly, a fully task-aware approach with $\lambdaforecast = 0$ yields poor predictions since it does not regularize for prediction errors. This motivates our weighted co-design approach on the left.}
\label{fig:forecasts_comparison}
\end{center}
\vskip -0.2in
\end{figure}

\begin{figure}[ht]
\vskip 0.2in
\begin{center}

\includegraphics[width=0.8\columnwidth]{figures/forecast_legend_short.pdf}

\subfigure{
{\includegraphics[width=0.4\columnwidth]{figures/appendix/iot/z_4/forecasts_no_task_aware_sample_15_time_54_signal_1.pdf}}
}
\subfigure{
{\includegraphics[width=0.4\columnwidth]{figures/appendix/iot/z_9/forecasts_no_task_aware_sample_15_time_54_signal_1.pdf}}
}
\subfigure{
{\includegraphics[width=0.4\columnwidth]{figures/appendix/iot/z_4/forecasts_no_task_aware_sample_15_time_54_signal_2.pdf}}
}
\subfigure{
{\includegraphics[width=0.4\columnwidth]{figures/appendix/iot/z_9/forecasts_no_task_aware_sample_15_time_54_signal_2.pdf}}
}

\caption{\textbf{Example forecasts (IoT sensors).}  Example forecasts at $t=40$ when $Z=4$ (left) and $Z=9$ (right). 
Clearly, the predictions are more accurate and smooth when $Z=9$. However, with a smaller bottleneck of $Z=4$ (left), we achieve near-optimal control performance since we capture task-relevant features with a coarse forecast that captures high-level, but salient, trends.}
\label{fig:iot_forecasts}
\end{center}
\vskip -0.2in
\end{figure}

\begin{figure}[ht]
\vskip 0.2in
\begin{center}

\includegraphics[width=0.8\columnwidth]{figures/forecast_legend_short.pdf}

\subfigure{
{\includegraphics[width=0.4\columnwidth]{figures/appendix/cell/z_4/forecasts_no_task_aware_sample_1_time_30_signal_1.pdf}}
}
\subfigure{
{\includegraphics[width=0.4\columnwidth]{figures/appendix/cell/z_9/forecasts_no_task_aware_sample_1_time_30_signal_1.pdf}}
}

\subfigure{
{\includegraphics[width=0.4\columnwidth]{figures/appendix/cell/z_4/forecasts_no_task_aware_sample_1_time_30_signal_3.pdf}}
}
\subfigure{
{\includegraphics[width=0.4\columnwidth]{figures/appendix/cell/z_9/forecasts_no_task_aware_sample_1_time_30_signal_3.pdf}}
}


\caption{\textbf{Example forecasts (taxi scheduling).}  Example forecasts at of at $t=16$ when $Z=4$ (left) and $Z=9$ (right). This scenario had the worst prediction errors since the cell data is highly stochastic.}
\label{fig:cell_forecasts}
\end{center}
\vskip -0.2in
\end{figure}

\begin{figure}[ht]
\vskip 0.2in
\begin{center}

\includegraphics[width=0.8\columnwidth]{figures/forecast_legend_short.pdf}

\subfigure{
{\includegraphics[width=0.4\columnwidth]{figures/appendix/pjm/z_4/forecasts_no_task_aware_sample_1_time_84_signal_0.pdf}}
}
\subfigure{
{\includegraphics[width=0.4\columnwidth]{figures/appendix/pjm/z_9/forecasts_no_task_aware_sample_1_time_84_signal_0.pdf}}
}
\subfigure{
{\includegraphics[width=0.4\columnwidth]{figures/appendix/pjm/z_4/forecasts_no_task_aware_sample_1_time_84_signal_3.pdf}}
}
\subfigure{
{\includegraphics[width=0.4\columnwidth]{figures/appendix/pjm/z_9/forecasts_no_task_aware_sample_1_time_84_signal_3.pdf}}
}

\caption{\textbf{Example forecasts (battery charging).}  Example forecasts at $t=84$ when $Z=4$ (left) and $Z=9$ (right). As before, the predictions are more accurate and smooth when $Z=9$. However, with a smaller bottleneck of $Z=4$ (left), we achieve near-optimal control performance since we capture task-relevant features with a coarse forecast that captures high-level, but salient, trends.}
\label{fig:pjm_forecasts}
\end{center}
\vskip -0.2in
\end{figure}


\textbf{The fully task-aware ($\lambdaforecast = 0$) scheme is good for control but poor for forecasting.} 

Fig. \ref{fig:forecasts_comparison} compares the time-domain forecasts given by task-agnostic/weighted scheme and task-aware scheme. Note that the timeseries starts at $t=-W+1<0$ because $s_{-W+1:0}$ is needed at $t=0$. While the task-agnostic and weighted scheme make reasonable forecasts, the task-aware scheme focuses solely on improving the task-relevant control and imposes no penalties on the forecasting error, leading to poor forecasts. This motivates our weighted approach which balances the control cost and forecasting error.

\textbf{Small $Z$ (e.g., $Z=4$) produces coarse forecasts, which are suitable for good control performance.} 

Fig. \ref{fig:iot_forecasts}, Fig. \ref{fig:cell_forecasts} and Fig. \ref{fig:pjm_forecasts} present the time-domain forecasts with different bottleneck dimensions $Z$ for IoT, taxi scheduling, and battery charging scenarios, respectively. In general, for small $Z$ (e.g., $Z=4$), the task-agnostic scheme makes noisy forecasts which provides room for our weighted scheme to improve the control cost by considering a task-relevant objective. For large $Z$ (e.g., $Z=9$) both the task-agnostic and weighted scheme make smooth forecasts\footnote{The trend is less prominent for the taxi scheduling scenario, because the cell demand itself is rapidly-changing and highly-stochastic.}.

\begin{figure}[ht]
\vskip 0.2in
\begin{center}
\includegraphics[width=0.8\columnwidth]{figures/evolution_legend.pdf}

\subfigure{
{\includegraphics[width=0.31\columnwidth]{figures/appendix/iot/z_4/state_evolution_sample_1_state_2.pdf}}
\label{fig_synthetic_state_evolution}
}
\subfigure{
{\includegraphics[width=0.31\columnwidth]{figures/appendix/cell/z_4/state_evolution_sample_1_state_2.pdf}}
}
\subfigure{
{\includegraphics[width=0.31\columnwidth]{figures/appendix/pjm/z_4/state_evolution_sample_1_state_2.pdf}}
\label{fig_synthetic_forecast_errors}
}

\caption{Example evolution of $x(2)$ when $Z = 4$, for IoT (top), taxi scheduling (middle) and battery charging (bottom) scenarios, respectively. Clearly, our co-design approach has state evolutions closer to the unrealizable optimal solution (black) which assumes \textit{perfect} forecasts.}
\label{fig:state_evolution}
\end{center}
\vskip -0.2in
\end{figure}

\textbf{The state evolution of our task-aware/weighted scheme is closer to the optimal trace.} 

Fig. \ref{fig:state_evolution} shows the example state evolution of $x(2)$ for the three scenarios. Importantly, the black trace corresponds to an \textit{unrealizable} baseline with the lowest cost since it assumes \textit{perfect} knowledge of $\bolds$ for the future $H$ steps. We can see that our task-aware and weighted scheme have state evolution traces closer to the optimal trace than the competing task-agnostic scheme. This further explains why task-aware and weighted schemes can yield a near-optimal cost for small $Z$ while the task-agnostic benchmark cannot.



\end{document}
